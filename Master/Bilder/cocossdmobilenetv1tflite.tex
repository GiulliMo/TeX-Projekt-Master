
\begin{figure}[H]
	\centering
	\begin{tikzpicture}[
	]
	\begin{axis}[
	width=12cm,
	height=7cm,
	axis y line*=left,
	ticklabel style={% gilt für x und y
		/pgf/number format/.cd,
		use comma,% Komma als Dezimaltrenner
		1000 sep = {}% keine Tausendertrennung 
	},
	xlabel={$\text{Recall } \textit{r(t)}$},
	ylabel={$\text{Precision } \textit{p(t)}$},
	axis x line*=bottom,
	xmin=0, xmax=1, 
	ymin=0, ymax=1.2,
	%every axis plot/.append style={line width=1.0pt}
	legend pos=north east,
	]
	
	\addlegendentry{Eigener Datensatz}; % legende1
	\addplot[gray, line width=0.8pt]  table [x=Step,y=Value,col sep=comma] from {Bilder/cocossdmobilenetv1tflite.csv};
	\addplot[black, line width=0.8pt]  table [x=Step,y=Value,col sep=comma] from {Bilder/cocossdmobilenetv1cocotest.csv};
	\addlegendimage{/pgfplots/refstyle=plotbrr};				
	\addlegendentry{\textit{COCO}-Datensatz} ;%legende 2;
	%\addlegendentry{plot 1}
	\end{axis}
	\end{tikzpicture}
	\caption{Gegenüberstellung der \textit{Precision} und \textit{Recall} Kurven eines quantisierten \textit{Tensorflow Lite SSD MobileNet V1} Netzes. Durch die schwarze Kurve wird das Testergebnis durch den \textit{COCO}-Datensatz gezeigt. Die graue Kurve zeigt das \textit{Precision} und \textit{Recall} Verhältnis für die Analyse des eigenen Datensatzes. Tensorflow starter}
	\label{fig: ssdmobilenetv1}
\end{figure}
