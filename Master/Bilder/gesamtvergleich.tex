
\begin{figure}[H]
	\centering
	\begin{tikzpicture}[
	]
	\begin{axis}[
	width=12cm,
	height=7cm,
	axis y line*=left,
	ticklabel style={% gilt für x und y
		/pgf/number format/.cd,
		use comma,% Komma als Dezimaltrenner
		1000 sep = {}% keine Tausendertrennung 
	},
	xlabel={$\text{Recall } \textit{r(t)}$},
	ylabel={$\text{Precision } \textit{p(t)}$},
	axis x line*=bottom,
	xmin=0.0, xmax=1.0, 
	ymin=0, ymax=1.2,
	%every axis plot/.append style={line width=1.0pt}
	%legend pos=north east,
	legend cell align={left},
	legend style={at={(0.5,1)},anchor=north} ,
	legend columns = 3,
	]
	
	\addlegendentry{\scriptsize V2 SSDLite (90)}; % legende1
	\addplot[gray!50, line width=1pt]  table [x=Step,y=Value,col sep=comma] from {Bilder/ssdmobilenetv2tflite.csv};
%	\addplot[line width=0.8pt]  table [x=Step,y=Value,col sep=comma] from {Bilder/hog.csv};
	\addplot[dashed,line width=1pt]  table [x=Step,y=Value,col sep=comma] from {Bilder/ownnetv2.csv};
	\addplot[gray!50,dashed,line width=1pt]  table [x=Step,y=Value,col sep=comma] from {Bilder/ownnetv2ssdlite.csv};
	\addplot[dotted,line width=1pt]  table [x=Step,y=Value,col sep=comma] from {Bilder/ownssdmobilenetv1.csv};
	\addplot[dash dot,line width=1pt]  table [x=Step,y=Value,col sep=comma] from {Bilder/cocossdmobilenetv1tflite.csv};
	\addplot[line width=1pt]  table [x=Step,y=Value,col sep=comma] from {Bilder/hog.csv};
	\addlegendimage{/pgfplots/refstyle=plotbrr};				
%	\addlegendentry{HoG \& SVM (H)} ;
	\addlegendentry{\scriptsize mod. V2 SSD (1)} ;%legende 2;
	\addlegendentry{\scriptsize mod. V2 SSDLite (1)} ;%legende 2;
	\addlegendentry{\scriptsize mod. V1 SSD (1)} ;%legende 2;
	\addlegendentry{\scriptsize quant. V1 SSD (90)} ;%legende 2;
	\addlegendentry{\scriptsize HOG \& SVM} ;%legende 2;
	%\addlegendentry{plot 1}
	\end{axis}
	\end{tikzpicture}
	\caption{\textit{Precision-Recall} Kurven aller \textit{MobileNet} Architekturen in Anwendung auf den eigenen Datensatz. Die Abkürzungen \textit{mod.} und \textit{quant.} stehen für \textit{modifiziert} und \textit{quantisiert}.}
	\label{fig: genauigkeitsvergleich}

	
\end{figure}
