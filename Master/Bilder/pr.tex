\newcolumntype{C}[1]{>{\centering\arraybackslash}p{#1}}
\newcolumntype{M}[1]{>{\centering\arraybackslash}m{#1}}
\begin{table}[H]
	\caption{Tabelle zum besseren Verständnis der Begriffe \textit{True Positive}, \textit{False Positive}, \textit{ False Negative} und \textit{True Negative}. In der jeweiligen Zelle wird ein Beispiel im praktischen Kontext der Personenerkennung genannt.}
	\begin{center}
		\begin{tabular}{|c|c|c|c|}
			\hline
			\multicolumn{2}{|c|}{\multirow{2}{*}{}}&\multicolumn{2}{c|}{Realität}\\ \cline{3-4}
			\multicolumn{2}{|c|}{}&Wahr&Falsch\\
			\hline
			\multirow{3.2}{*}{Ausgabe}&Wahr&\makecell{\textit{True Positive} (TP)\\ Realität: Person vorhanden\\ Ausgabe: Person vorhanden}& \makecell{\textit{False Positive} (FP) \\ Realität: Keine Person  \\ Ausgabe: Person vorhanden}\\ \cline{2-4}
			&Falsch&\makecell{\textit{False Negative} (FN)\\ Realität: Person vorhanden\\ Ausgabe: Keine Person}&\makecell{\textit{True Negative} (FN)\\ Realität: Keine Person\\ Ausgabe: Keine Person}\\
			\hline
		\end{tabular}
	\end{center}

	\label{fig: pr}
\end{table}