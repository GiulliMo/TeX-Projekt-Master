\begin{figure}[H]
	\centering

\begin{tikzpicture}[round/.style={rounded corners=1.5mm,minimum width=3cm,inner sep=2mm,above right,draw,align=left}]

\node[round] (stop) at (1,10) {Stop 0};
\node[round] (wait) at (1,8) {Warten 1};
\node[round] (localization) at (3.5,6) {Statische Karte 3};
\node[round] (slam) at (-1.5,6) {SLAM 2};
\node[] (autonomname) at (5,2) {Autonom 6};
\node[round] (explore) at (6,0) {Ziel 8};
\node[round,align=center] (goal) at (1,0) {Erkunden 7};
\node[round,above,fit=(explore)(goal)(autonomname)] (autonom) {};
\node[round,align=center] (manual) at (-4,0) {Manuell 5};
\node[] (drivename) at (2.5,4) {Fahren 4};
\node[round,fit=(manual)(autonom)(drivename)] (drive) {};

\draw[-latex] ($(stop.north) + (-0,0.5)$) coordinate (temp) to (stop);
\fill (temp) circle (0.1);
\draw[-latex]  (stop) to (wait);
\draw[-latex,bend left]  (wait.333) to (localization.north);
\draw[-latex,bend right]  (wait.207) to (slam.north);
\draw[-latex,bend left]  (slam.south) to (drive.north);
\draw[-latex,bend right]  (localization.south) to (drive.north);
\draw[-latex,bend right] ($(autonom.north) + (-2.5,1)$) coordinate (temp2) to (manual.north);
\draw[-latex,bend left] ($(autonom.north) + (-2.5,1)$) coordinate (temp2) to (autonom.110);
\fill (temp2) circle (0.1);
\draw[-latex,bend right] ($(goal.north) + (2.5,0.8)$) coordinate (temp3) to (goal.north east);
\draw[-latex,bend left] ($(goal.north) + (2.5,0.8)$) coordinate (temp3) to (explore.north west);
\fill (temp3) circle (0.1);
%\draw[-latex, bend left] ($(neutral.north east) + (-0.5,0.3)$) coordinate (temp3) to (neutral);
%\fill (temp3) circle (0.1);
%\draw[-latex, bend left] (left) to (right);
%\draw[-latex, bend left] (right) to (left); 
%\draw[-latex, bend left] (left) to (neutral);
%\draw[-latex, bend left] (neutral) to (right);
%\draw[-latex, bend left] (neutral) to (left);
%\draw[-latex, bend left] (right) to (neutral);
%
%\draw[-latex, bend left] (pause) to (running);
%\draw[-latex, bend left] (pause) to[in=-135,out=-90] (ident);
%\draw[-latex, bend left] (pause) to (observe);
%\draw[-latex, bend left] (observe) to (origin);
%%\draw[-latex, bend left] (origin) to (origin);
%\draw[-latex, bend left] (origin) to (pause);
%\draw[-latex, bend left] (running) to (origin);
%\draw[-latex, bend left] (ident) to[out=-60,in=-90] (origin);
%\draw[-latex, bend left] (rotLeft) to (rotRight);
%\draw[-latex, bend left] (rotRight) to (rotLeft);
\end{tikzpicture}
\caption{Konzept eines hierarchischen Zustandsautomaten in Form eines UML-Diagramms. Zustände sind als Boxen mit Abrundungen dargstellt. Ineinander verschachtelte Zustände erben Funktionen von dem jeweils größeren Zustand. Pfeile deuten mögliche Transitionen an. Startzustände werden durch schwarze Punkte gezeigt.}
\label{fig: uml}
\end{figure}