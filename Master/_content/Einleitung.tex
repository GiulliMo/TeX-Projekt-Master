
\chapter{Einleitung}
\label{ch: Einleitung}
	

		Das Thema der künstlichen Intelligenz (KI) dringt zunehmend in den Alltag des Menschen ein. \textit{Smart Home} Geräte wie \textit{Amazons Alexa} oder der \textit{Google Assistant} wurden bereits Ende des Jahres 2018 in Anzahlen jenseits der 20 Millionen verkauft \cite{smart}. Derartige Technologien begleiten den Menschen jedoch nicht im Alltag, sondern auch in der Transport- und Logistikbranche. Eine Potenzialanalyse zur künstlichen Intelligenz der Firma \textit{Sopra Steria} zeigt, dass bereits im Jahr 2017 20\% aller befragten Unternehmen derartige Systeme einsetzten \cite{sopra}. 37\% planten den zukünftigen Einsatz \cite{sopra}. Die Implementierung solcher Systeme hat Einfluss auf verschiedene Eigenschaften der Wertschöpfungskette \cite{sopra}. Einige der zur Logistikbranche gehörenden Transportfahrzeuge sind ebenfalls mit KI ausgestattet \cite{sopra}.\\
		
		\begin{figure}[H]
			\begin{minipage}[b]{0.49\textwidth}
				(a)
				\includegraphics[width=0.9\textwidth]{Bilder/oben.pdf}
			\end{minipage}
			\begin{minipage}[b]{0.49\textwidth}
				(b)
				\includegraphics[width=0.9\textwidth]{Bilder/seite.pdf}
			\end{minipage}
			\centering
			\caption{(a) Darstellung des autonomen Logistik-Fahrzeugs aus der Draufsicht. Die acht benachbarten Rechtecke in der Mitte des Fahrzeugs stellen die Ladefläche des Fahrzeugs dar. (b) Darstellung aus der Seitenansicht. Oben links in der Abbildung ist der Schaltschrank zu sehen. Die Räder des Fahrzeugs befinden sich unten links und unten rechts in der Darstellung.}
			\label{fig: Darstellung des ALFs}
		\end{figure}
		
		Das autonome Logistikfahrzeug ALF ist ein solches Transportfahrzeug. Die Idee des ALFs ist es, ein Fahrzeug zu entwickeln, das nach seiner Fertigstellung Logistikaufgaben am Standort der Hochschule Bochum löst. Es dient im Labor für Antriebstechnik der Hochschule Bochum als Versuchs- und Entwicklungsplattform für praktische Anwendungen. Der Entwicklungsprozess stellt sich aus diversen Bachelor- und Masterarbeiten zusammen, die sowohl Hardware als auch Softwareintegrationen vorsehen.\\
		
		Bisher wurden zwei Abschlussarbeiten inklusive der praktischen Anwendung am ALF geschrieben. M.Sc. Dennis Hotze und M.Sc. Dominik Eickmann entwickelten in ihrem Masterprojekt das Fahrzeug und konnten Fahraufgaben ferngesteuert und manuell erledigen \cite{alf}. Während der darauffolgenden Bachelorarbeit wurde eine Schlupfkompensation entwickelt, die den Drift am Fahrzeug durch Eingabe von Umgebungsinformationen verhindert \cite{Bachelorarbeit}. Weiterhin wurden Funktionen entwickelt, um grundlegende und autonome Fahraufgaben zu lösen \cite{Bachelorarbeit}. Das autonome Logistikfahrzeug aus der vorangegangenen Bachelorarbeit dient auch in dieser Masterarbeit als Versuchsplattform.\\
		
		Der tägliche Kontakt zu Menschen ist nicht nur für das ALF, sondern auch für andere Transportfahrzeuge im öffentlichen Raum häufig unentbehrlich. Kreuzen sich die Wege eines autonomen Fahrzeugs mit der einer oder mehrerer Personen, darf es in keiner Weise zur Kollision kommen. Das ALF zählt mit einem Maximalgewicht von 600 kg zu der Art von Fahrzeugen, die bei einem Zusammenstoß Verletzungen hervorrufen können. Ein solches System muss folglich in der Lage sein, bei einer Annäherung von Personen eine gesonderte Gefahreneinschätzung vorzunehmen.\\
		
		Im Einsatz an der Hochschule Bochum kommt es eher selten zu Begegnungen mit Tieren, die nicht in Begleitung von Personen sind. Dementsprechend wird das System im Rahmen dieses Projekts auf die Erkennung von Personen beschränkt. Ein weiteres, übliches Szenario solcher Systeme im öffentlichen Raum ist die Bedienung durch nicht autorisierte Personen. Für derartige Zwecke muss ein Mensch nicht nur als bewegtes Objekt, sondern auch als solcher erkannt werden. \\
		
		Das Hauptziel dieser Masterarbeit ist die Entwicklung einer Personenerkennung. Die integrierten \textit{Kinect}-Kameras dienen neben einem \textit{Light Detection and Ranging} (LiDAR) Sensor als einzige optische Sensoren des autonomen Logistikfahrzeugs. Die Sensordaten werden nach wie vor mithilfe des \textit{Robot Operating System} (ROS) verarbeitet \cite{Bachelorarbeit}. Zwar gibt es technische Lösungen für eine Personenerkennung mithilfe von zweidimensionalen Laserdaten, jedoch nutzen state-of-the-art Systeme Bildverarbeitung zur Erkennung von Objekten \cite{mobilenets, hogsvm}. Folglich wird das System zur Personenerkennung mithilfe der beiden \textit{Kinect}-Kameras ausgelegt. Weiterhin wird das System nicht nur Personen detektieren können, sondern diese auch wiedererkennen. \\
		
		Während der Bearbeitung von Transport- oder Fahraufgaben eines autonomen Fahrzeugs kann es zu diversen Komplikationen kommen. Beispielsweise können besonders im Anwendungsbereich der Hochschule Bochum Objekte den Verlauf einer Route unterbrechen und ein Ziel sogar unerreichbar machen. Häufig können diese Probleme durch menschliche Hilfe beseitigt werden. Wiederum setzt dies eine Interaktion mit umstehenden Personen voraus. Eine Besonderheit dieser Masterarbeit ist die theoretische und praktische Entwicklung parallel zu einem weiteren Projekt. Hannes Dittmann entwickelt in seiner Masterarbeit eine Sprachverarbeitung zur Klassifikation anwendungsorientierter Sprache \cite{Dittmann}. Diese stellt eine auditive \textit{Mensch-Maschine-Interface} (MMI) Schnittstelle her. So kann eine Person über ein Aufnahmegerät mit dem System kommunizieren. Die klassifizierte Sprache ist ohne Weiteres nicht in der Lage, das Fahrzeug enstprechend zu steuern. Demnach wird in dieser Masterarbeit ein Konzept zur Steuerung des ALFs mit klassifizierter Sprache entwickelt und am Fahrzeug implementiert.\\
		
		Die Struktur dieser Masterthesis ist in 4 Kapiteln gegliedert. Beginnend mit Kapitel \ref{ch: Grundlagen} werden die Grundlagen der eingesetzten Methoden und Systeme vermittelt. Hierbei werden Grundbegriffe des Themenbereichs der künstlichen Intelligenz und insbesondere der neuronalen Netze erklärt. Kapitel \ref{ch: Konzeptionierung} zeigt, wie die vermittelten Grundlagen zu einem System zusammengeführt und verwendet werden. Eine Evaluierung der im Feld getesteten Personenerkennung ist in Kapitel \ref{ch: Verifikation} beschrieben. Ein Vergleich alternativer Lösungen ist dort ebenfalls präsentiert. Das letzte Kapitel beschreibt zusammenfassend die gesamte Masterarbeit und gibt einen Ausblick für zukünftige Projekte am autonomen Logistikfahrzeug. \\
		
		Diese und die Masterarbeit von \textit{Hannes Dittmann} bilden im praktischen Kontext ein überarbeitetes Gesamtsystem des bereits bestehenden autonomen Logistikfahrzeugs \cite{Dittmann}. Die genannten Ziele sind im angehängten Lastenheft \ref{it: Lastenheft} festgehalten. Die erarbeiteten Ergebnisse werden so anhand eines Verifikationsplans am Fahrzeug geprüft.
		
		
		
		
	