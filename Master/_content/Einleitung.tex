
\chapter{Einleitung}
\label{ch: Einleitung}
	
	\section{Motivation}
	\label{sec: Motivation}
		Das Thema der künstlichen Intelligenz (KI) ist heutzutage allgegenwärtig. Smart Home Geräte wie Amazons Alexa, Siri der Firma Apple oder der Google Assistant gehören mittlerweile in jeden ... deutschen Haushalt und enthalten KI zur Spracherkennung. Derartige Technologien begleiten den Menschen jedoch nicht nur Zuhause sondern auch in der Transport- und Logistikbranche. Eine Potenzialanalyse zur künstlichen Intelligenz der Firma Sopra Steria zeigt, dass bereits im Jahr 2017 20\% aller befragten Unternehmen solche Systeme einsetzten. 37\% planten den zukünftigen Einsatz. Die Implementierung solcher Systeme hat Einfluss auf verschiedenste Eigenschaften der Wertschöpfungskette. Die Qualität des Fachprozesses wird mit ebenfalls steigender Geschwindigkeit erhöht. Die zur Logistikbranche gehörenden Transportfahrzeuge sind ebenfalls mit KI ausgestattet und sorgen so für weniger Arbeitsunfälle und eine schnellere, präzisiere Abarbeitung der Logistikaufgaben.\\
		
		
		Das Projekt dieser Masterarbeit wird praktisch am autonomen Logistikfahrzeug angewendet, das aus dem Labor für Antriebstechnik der Hochschule Bochum stammt. Die Idee des ALFs ist es ein Fahrzeug zu entwickeln, das nach seiner Fertigstellung Logistikaufgaben am Standort der Hochschule Bochum lösen soll. Der Entwicklungsprozess stellt sich aus diversen Bachelor- und Masterarbeiten zusammen, die sowohl Hardware, als auch Softwareimplementierungen vorsehen. Bisher wurden zwei Abschlussarbeiten inklusive der praktischen Anwendung am ALF geschrieben. M.Sc. Dennis Hotze und M.Sc. Dominik Eickmann entwickelten in ihrem Masterprojekt das Fahrzeug und konnten Fahraufgaben ferngesteuert und manuell erledigen. Während der darauffolgenden Bachelorarbeit wurde eine Schlupfkompensation entwickelt, die den Drift am Fahrzeug durch Eingabe von Umgebungsinformationen verhindert. Weiterhin wurden Funktionen entwickelt, um grundlegende und autonome Fahraufgaben zu lösen. Das autonome Logistikfahrzeug aus der vorangegangenen Bachelorarbeit dient auch in dieser Masterarbeit als Versuchsplattform. 
	
	\section{Zielsetzung des Projekts}
		Die Grundidee und Herausforderung dieser Masterarbeit ist die Interaktion zwischen Menschen und Roboter und der dadurch resultierenden Bedienung des Systems ohne Eingabegerät. Bisher wurden anzufahrende Posen per Mausklick eingegeben oder Fahrmodi manuell gewechselt. Der Informationsfluss wird hierbei rein visuell und akustisch passieren. Letzteres wird in der Masterarbeit von Herrn Dittmann behandelt und durch ausgewählte Schnittstellen mit diesem Projekt verknüpft, um ein Gesamtsystem zu bilden. Die visuelle Komponente kann im weiteren Entwicklungsprozess des Roboters für verschiedene Anwendungsbereiche genutzt werden. Ziel dieser Masterarbeit ist die Erkennung und Unterscheidung von Personen. Das System wird zwischen bekannten und unbekannten Personen unterscheiden können und Informationen aus den gegebenen Daten generieren, die in der weiteren Entwicklung nützlich sind. 
	