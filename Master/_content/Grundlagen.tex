\chapter{Grundlagen}
\label{ch: Grundlagen}
	Für ein besseres Verständnis, der in Kapitel Konzept... angewandten Methoden, werden anbei die Grundlagen behandelt. Informationen zu der verwendeten Hard- und Software wurden bereits in der vorangegangenen Bachelorarbeit vermittelt. Aufgrund Anforderung A... ist während der praktischen Anwendung keine Änderung der Hardware vorgesehen.
 
	
 	\section{Eigenschaften von neuronalen Netzen}
	\label{sec: ROS}
	
	Das Neuronennetz des menschlichen Gehirns dient als Vorbild für künstliche, neuronale Netze (KNN). Diese werden heutzutage als Lösung diverser Anwendungsprobleme angewendet, in denen komplexe Strukturen und Muster aus großen Datenmengen erkannt werden sollen. Das in diesem Projekt zugrundeliegende Bildverarbeitungsproblem besitzt die beschriebenen Eigenschaften und eignet sich somit für den Einsatz zur Erkennung von Personen. Anders als bei den meisten programmierten Applikationen ist die Ausgabe von KNN's lediglich probabilistisch. Bei dem vorliegenden, autonomen Logistikfahrzeug werden zur Personenerkennung derartige neuronale Netze verwendet.
	
		\subsection{Anwendung von neuronalen Netzen}
		
			Das Grundlage für die Eingabe in ein neuronales Netz ist die Skalierung der vorliegenden Daten auf eine definierte Größe. Diese wäre beispielsweise bei einem Anwendungsfall mit einer Audiospur die Frequenzspektren oder bei einem Bildverarbeitungsproblem die Pixel eines Bildes. Die skalierten Daten werden in einem Tensor gegeben der die Dimensionen der Eingabe hat. Somit unterteilt sich ein Bild in die drei Dimensionen, die Höhe, die Weite und die Farbwerte der Primärfarben pro Pixel. 
	
		\subsection{Lernprozess}
			%gewichte stellen sich ein
			%train und val datensatz
		
		\subsection{Unterscheidung verschiedener Netze}
		
		\subsection{Evaluation neuronaler Netze}
			
			Die Ausgabe von neuronalen Netzen ist probabilistisch und nicht vorhersehbar. Folglich bestehen diverse Metriken für Evaluationen, die derartige Systeme messbar machen. 
		
	\section{Objekterkennung}
	\label{sec: Mecanumräder}
	Bei der visuellen Objekterkennung wird ein Objekt, das auf einem Bild gezeigt ist, mit einer gewissen Wahrscheinlichkeit inklusive der Position in der Abbildung erkannt. Die drei Abstraktionsebenen einer solchen Erkennung unterteilen sich in Bildklassifikation, Objektlokalisierung und semantische Segmentierung ...2014Bild. Letzteres kommt in dieser Arbeit nicht zur Anwendung und wird aufgrunddessen im Folgenden nicht behandelt. Weiterhin wird ebenfalls die Objekterkennung durch neuronale Netze betrachtet und in Kapitel... verglichen.
	
		\subsection{Bildklassifikation}
		Die Bildklassifikation beschreibt eine Zuweisung von Objektkategorien zu einem gegebenen Bild. Mithilfe einer Merkmalsextraktion werden Merkmalsvektoren extrahiert und können so in einem Klassifikator berechnet werden. Ein Gängiges Verfahren zur Merkmalsgewinnung ist das sogenannte Histogram of oriented gradients (HOG). Bei diesem Verfahren werden in einem Bild auftretende Intensitäten geprüft und so Kanten und Ecken als Histogramm gespeichert. Die Support Vektor Machine ist eine typische Methode zur Klassifikation. Es handelt sich hierbei um ein Verfahren, das Klassen durch eine sogenannte Hyperebene voneinander trennt. Diese Methode hat sich vor allem aufgrund ihrer kurzen Rechenzeit durchgesetzt.
			
		\subsection{Objektlokalisierung}
		Wie in Kapitel... beschrieben, soll die Augabe einer Objekterkennung auch den Ort eines Objektes enthalten. Für jedes erkannte Objekt wird ein Rechteck in Form von Pixelkoordinaten erzeugt, das den Interessensbereich beschreibt.
	
		\subsection{Objekterkennung durch neuronale Netze}
		
	

			
	\section{Zustandsautomat}
	\label{sec: Zustandautomat}
	Die Idee der Nutzung eines Zustandsautomaten oder auch endlicher Automat (EA) ergab sich im Laufe der Entwicklungsphase. In der in Kapitel ... erwähnten Bachelorarbeit werden diverse Modi beschrieben, die den Aufruf von unterschiedlichen ROS-Knoten vorraussetzen. Aufgrund der Analogie zwischen den beschriebenen Modi und der Zustände eines Zustandsautomaten wird die Nutzung eines solchen Automaten begründet. Im Folgenden wird auf die Eigenschaften eines endlichen Automats eingegangen.\\
	
	Im Allgemeinen geht es bei einem Zustandsautomaten um die Beschreibung der Zustände (engl. States) eines Objekts. Dabei stellt das Objekt meist das Gesamtsystem dar, etwa ein Getränkeautomat oder wie dieser Arbeit autonomes Fahrzeug. States sind durch Bedingungen verknüpft und lösen während sogenannter Ereignisse eine Transition aus, die den Wechsel des Zustands ausübt. Weiterhin Bilden die Zustände in ihrer Gesamtheit den Lebenszyklus des Objekts. Ein Getränkeautomat befindet sich bekanntermaßen beim Eintreffen eines Kunden in einer Art Bereitschaft. Übertragen auf die Theorie eines Zustandsautomaten wäre dies ein Bereitschaftszustand. Die Auswahl des Getränks und die Eingabe des entsprechenden Geldbetrags können beispielhaft als Ereignisse interpretiert werden. Somit wird ein Transition durchgeführt und der Zustand der Getränkeausgabe wird losgetreten. Wurde das Getränk ausgegeben und entnommen, geschieht der Wechsel in den Bereitschaftszustand und der beschriebene Zyklus ist komplettiert.\\
	
	Seit dem Bestehen der endlichen Automaten haben sich in der Praxis zwei Typen durchgesetzt. Mealy und Moore Automaten unterscheiden sich grundlegend in ihrem Verhalten und können durch folgende Gleichungen beschrieben werden.\\
	
	Gleichung Mealy...\\
	
	In den Gleichung ... beschreibt $\varphi$ die Transitionsfunktion und $\psi$ die Ausgabefunktion des Mooreautomats. Die Transition steht in Abhängigkeit von $\zeta_t $, die aktuelle Eingabe, und $\alpha_t$, der aktuelle Zustand selbst. Mithilfe der Transitionsfunktion lässt sich der Zustand bestimmen, der im folgenden Zeitschritt $t$ angestrebt werden soll. Der Ausgang des Moore Automaten wird durch die Ausgangsfunktion $psiup$ berechnet. Diese hängt genau wie die Transitionsfunktion von der Eingabe und dem Zustand zum Zeitpunkt $t$ ab. \\
	
	Gleichuing Moore...\\
	
	Beim Vergleich der beiden Gleichungen ... und ... stellt sich heraus, dass die Ausgangsfunktion $psi$ in der Beschreibung des Verhaltens eines endlichen Automaten durch Moore lediglich vom Ausgang zum Zeitpunkt $t$ abhängig ist. Zur Veranschaulichung 
	
	Eine Unterkategorie der Finiten Automaten ist der Hierarchische Zustandsautomat. Die Besonderheit hierbei ist die Zusammensetzung aller vorangegangenen Zustände eines aktiven Zustands. Diese sind bei der hier beschriebenen hierarchisch aufgebauten Maschine nämlich ebenfalls aktiv. So besteht die Möglichkeit eines aufeinander aufbauenden Endzustands. 
	
	
		
		
	\section{Bestimmung von Positionskoordinaten}
		Während der Durchführung autonomer Fahr- bzw. Logistikaufgaben können diverse Probleme auftreten, die eine erfolgreiche Bearbeitung verhindern können. Beispielsweise können Türen geschlossen sein oder Gegenstände die geplante Route blockieren. Da das ALF nicht über die technischen Möglichkeiten besitzt derartige Problemstellungen zu lösen, müssen Menschen Abhilfe schaffen. Für diese Zwecke ist die Kenntnis über die letzte Position der erfassten Personen realtiv zur statischen Karte  notwendig. Anstehende Fahraufgaben werden, bedingt durch das Vorgängerprojekt, mithilfe des Robot Operating Systems gelöst. Personen können folglich als Position in das ROS Netzwerk veröffentlicht. Dies ermöglicht dem Roboter die veröffentlichten Positionen anzufahren. Die Eintragung der Position in die statische Karte setzt die Beschreibung der Position als Koordinaten vorraus. Für die Bestimmung der Positionskoordinaten wird ein zweidimensionales Bild und die dazugehörigen Tiefeninformationen genutzt. Die Koordinate xloc beschreibt hier die longitudinale Entfernung von der Kamera zur Person. Eingehende laterale Distanzen werden durch die Koordinate yloc dargestellt.
	

