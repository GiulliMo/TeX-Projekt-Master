\chapter{Grundlagen}
\label{ch: Grundlagen}
	
 
	
 	\section{Eigenschaften von neuronalen Netzen}
	\label{sec: ROS}
	
	Das Neuronennetz des menschlichen Gehirns dient als Vorbild für künstliche, neuronale Netze (KNN). Diese werden heutzutage als Lösung diverser Anwendungsprobleme angewendet, in denen komplexe Strukturen und Muster aus großen Datenmengen erkannt werden sollen. Das in diesem Projekt zugrundeliegende Bildverarbeitungsproblem besitzt die beschriebenen Eigenschaften und eignet sich somit für den Einsatz zur Erkennung von Personen. Anders als bei den meisten programmierten Applikationen ist die Ausgabe von KNN's lediglich probabilistisch. Bei dem vorliegenden, autonomen Logistikfahrzeug werden zur Personenerkennung derartige neuronale Netze verwendet.
	
		\subsection{Anwendung von neuronalen Netzen}
		
			Das Grundlage für die Eingabe in ein neuronales Netz ist die Skalierung der vorliegenden Daten auf eine definierte Größe. Diese wäre beispielsweise bei einem Anwendungsfall mit einer Audiospur die Frequenzspektren oder bei einem Bildverarbeitungsproblem die Pixel eines Bildes. Die skalierten Daten werden in einem Tensor gegeben der die Dimensionen der Eingabe hat. Somit unterteilt sich ein Bild in die drei Dimensionen, die Höhe, die Weite und die Farbwerte der Primärfarben pro Pixel. 
	
		\subsection{Lernprozess}
			%gewichte stellen sich ein
			%train und val datensatz
		
		\subsection{Unterscheidung verschiedener Netze}
		
		\subsection{Evaluation neuronaler Netze}
			
			Die Ausgabe von neuronalen Netzen ist probabilistisch und nicht vorhersehbar. Folglich bestehen diverse Metriken für Evaluationen, die derartige Systeme messbar machen.  
		
	\section{Objekterkennung durch neuronale Netze}
	\label{sec: Mecanumräder}
		
		\subsection{Datensätze}
		
	

			
	\section{Statemachine}
	\label{sec: Regelung}
	
	
		
		
	\section{Bestimmung von Positionskoordinaten}
	
	

