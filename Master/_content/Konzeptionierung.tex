\chapter{Konzeptionierung}
\label{ch: Konzeptionierung}

	
	
	\section{Anforderungserhebung mit CONSENS}
	\label{sec: Anforderungserhebung}
			
	
	\section{Konzept und Aufbau der Personenerkennung}
	
		\subsection{Wirkstruktur der Personenerkennung}
		
		\subsection{Auswahl und Training der verwendeten neuronalen Netze}
		
		\subsection{Schnittstelle zwischen Python und ROS}
		
		\subsection{Erstellung von Objektinformationen}
	
	\section{Funktionsweise des Gesamtsystems}
	Der Kern dieser Arbeit ist die Personenerkennung im praktischen Kontext des im Kapitel ... beschriebenen autonomen Logistikfahrzeugs. Aufgrund dessen sind einzelne Programme nicht als abgeschlossenes System zu betrachten. In Kapitel .../consens wurden bereits alle Schnittstellen zu verbauten Hardware- und Softwarekomponenten präsentiert. Das vollständige System der Personenerkennung und der Aufbau des entwickelten, endlichen Automats wird im folgenden Kapitel erklärt.\\
	
	Die Personenerkennung am ALF wird mithilfe der Bildinformationen von zwei \textit{Kinect}-Kameras betrieben. In Kapitel.../rospython wurden bereits zwei Lösungsansätze in der Softwarentwicklung in Zusammenspiel mit ROS und Python präsentiert. Aufgrund einer starken Belastung der im Roboter verbauten Recheneinheit während der parallelen Bildverarbeitung beider eingehender Bilder, wurde das Programm auf eine serielle Verarbeitung umgestellt. Der Befehl der parallelen Verarbeitung erzwingt Berechnungsprozesse mit derselben Frequenz, die durch den Eingang der Bilder beider Kameras vorgegeben wird. Mithilfe der seriellen Abarbeitung  war es ebenfalls möglich die Häufigkeit der Berechnungen zu steuern und damit verbundene Programmoptimierungen vorzunehmen. Auslastungen des verbauten Computers können so eingespart werden und für weitere, parallel laufende Prozesse genutzt werden. Dies wird zum Beispiel durch eine gezielte Verzögerung der Personenerkennung erreicht. Hierbei werden Pausen mit der gewünschten Dauer zwischen Bildverarbeitungsprozessen eingelegt bis ein relevantes Bild erkannt wird. Erst dann arbeitet die Personenerkennung mit der maximalen Geschwindigkeit. Als relevant werden Bilder eingestuft, die eine Person enthalten.  
	\newpage
	\begin{figure}[H]
	\begin{tikzpicture}[node distance = 2cm, auto]
		
		% Place nodes
		\node [papStart] (Start1){Start};
		\node [papProcess, below of = Start1,label={[shift={(5,-0.6)}]\footnotesize\textit{Aktuelles Bild wird mit KNN analysiert}}] (pro1){Prozess};
		\node [papDecision, below of = pro1, yshift= -9mm,label={[shift={(2.7,-0.6)}]\footnotesize\textit{Menschen im Bild?}}](dec1){Entscheidung};
		\node [papProcess, right of = dec1,xshift=25mm,label={[shift={(5,-0.6)}]\footnotesize\textit{Gesichter im Interessensbereich detektieren}}](pro3){Prozess};
		\node [papDecision, below of = pro3, yshift= -9mm,label={[shift={(5,-0.6)}]\footnotesize\textit{Gesicht im Interessenbereich?}}](dec2){Entscheidung};
		\node [papProcess, below of = dec2, yshift= -9mm,label={[shift={(5,-0.6)}]\footnotesize\textit{Merkmalsextraktion des Gesichts}}](pro4){Prozess};
		\node [papDecision, below of = pro4, yshift= -9mm,label={[shift={(5,-0.6)}]\footnotesize\textit{Gesicht bekannt?}}](dec3){Entscheidung};
		\node [papDecision, below of = dec3, yshift= -18mm,label={[shift={(5,-0.6)}]\footnotesize\textit{Dasselbe unbekannte Gesicht oft genug hintereinander erkannt?}}](dec4){Entscheidung};
		\node [papProcess, below of = dec4, yshift= -9mm,label={[shift={(5,-0.6)}]\footnotesize\textit{Eigenschaften des Objekts vom Typ Mensch aktualisieren}}](pro5){Prozess};
		%\node [papData, right of = dec3, xshift= 25mm,label={[shift={(5,-0.6)}]\footnotesize\textit{Gesicht bekannt?}}](dat1){I/O};
		\node [papEnd, below of = dec1, yshift= -40mm] (End) {Ende};
		
		% Place joins
		\coordinate [below of = dec1, yshift= -9mm] (join1);
		
		% Draw edges
		\path [papLine] (Start1) -- (pro1);
		\path [papLine] (pro1) -- (dec1);
		\path [papLine] (dec1) -- node [above] {\papYes} (pro3);
		\draw (dec1) -- node [right] {\papNo} (join1);
		\path [papLine] (pro3) -- (dec2);
	%	\path [papLine] (dec2) -- node [above] {\papYes} (pro4)
		\draw (dec2) -- node [above] {\papNo} (join1);
		
		\path [papLine] (join1) -- (End);
		
	\end{tikzpicture}
	\end{figure}
	
		%vektor statemachine
	
		
		
				   		

