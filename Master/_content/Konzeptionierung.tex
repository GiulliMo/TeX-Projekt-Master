\chapter{Konzeptionierung}
\label{ch: Konzeptionierung}

	
	
	\section{Anforderungserhebung mit CONSENS}
	\label{sec: Anforderungserhebung}
	
			Die Anforderungen an den ALF werden unter anderem mit Hilfe der Umfeldmodellierung aus der Conceptual design specification technique for the engineering of complex Systems (CONSENS) erhoben und als Anforderungsliste im Lastenheft festgehalten. Die CONSENS Methode wird zur systematischen Spezifikation von komplexen Systemen angewendet und  an der Hochschule Bochum am Institut für Systemtechnik geschult. Im Umfeldmodell der Bachelorarbeit, das in Abbildung \ref{fig: umfeldmodell} zu sehen ist, werden die Komponenten behandelt, um die Schlupfregelung und  SLAM-Kartografierung umzusetzen.
			
	
	\section{Konzept und Aufbau der Personenerkennung}
	
		\subsection{Wirkstruktur der Personenerkennung}
		
		\subsection{Auswahl und Training der verwendeten neuronalen Netze}
		
		\subsection{Schnittstelle zwischen Python und ROS}
		
		\subsection{Erstellung von Objektinformationen}
	
	\section{Funktionsweise des Gesamtsystems}
	
	
		
		
				   		

