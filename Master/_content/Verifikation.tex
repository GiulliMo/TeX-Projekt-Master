\chapter{Evaluation}
\label{ch: Verifikation}
Zur Evaluation der künstlichen neuronalen Netze wird eine anwendungsorientierte \textit{Benchmark} durchgeführt. Hierbei wird anhand der in Kapitel \ref{subsec: Entwickeltes neuronales Netz} beschriebenen Datensätze die \textit{Precision} und \textit{Recall} Methode angewendet. Weiterhin werden die Benchmarks auf einer Grafikkarte, dem integrierten Computer des ALFs und einem eingebettem System ausgeführt. Die Eckdaten des Grafikkarte sowie des ALFs sind in Anhang \ref{a: } präsentiert. Als eingebettetes System wird ein \textit{Raspberry Pi 3 Model B+} verwendet. Es ist keine Veränderung der Genauigkeit je Netz auf den jeweiligen Geräten zu erwarten. Jedoch können so die Bearbeitungszeiten pro Bild für unterschiedliche Hardware verglichen werden. \\

Insgesamt werden für den Test durch \textit{COCO-Datensatz} 12755 Bilder aus dem Trainingsdatensatz verwendet. Der technische Hintergrund hierfür ist in Kapitel \ref{subsec: Entwickeltes neuronales Netz} zu finden. Als Vergleich analysiert jedes Netz auch den eigenen Datensatz. So kann die Performance am Einsatzort des ALFs an der Hochschule Bochum evaluiert werden. Jedes Bild wird für die verwendeten, neuronalen Netze auf eine Pixelgröße von $300 \times 300$ skaliert. Für die Evaluation des HoGs wird eine höhere Auflösung gewählt. Hierbei wird eine Seite des Bildes softwareseitig auf 400 Pixel begrenzt.\\

Der in Grundlagenkapitel \ref{subsec: evaluation neuronaler netze} beschriebene mAP-Wert wird häufig auf Objekterkennungssystemen mit multiplen Klassen angewendet. Die hier entwickelte Personenerkennung soll jedoch lediglich die Klasse \textit{Person} erkennen. Somit ist der mAP-Wert in diesem Fall der Mittelwert eines Messwerts und kann als Integral der \textit{Precision-Recall}-Kurve angesehen werden. Im Verlauf der Evaluation der angewendeten Systeme wird mithilfe einer Berechnungssoftware jeweils der mAP-Wert berechnet.
 

\begin{figure}[H]
	\centering
	\begin{tikzpicture}[
	]
	\begin{axis}[
	width=12cm,
	height=7cm,
	axis y line*=left,
	ticklabel style={% gilt für x und y
		/pgf/number format/.cd,
		use comma,% Komma als Dezimaltrenner
		1000 sep = {}% keine Tausendertrennung 
	},
	xlabel={$\text{Recall } \textit{r(t)}$},
	ylabel={$\text{Precision } \textit{p(t)}$},
	axis x line*=bottom,
	xmin=0, xmax=0.3, 
	ymin=0, ymax=0.3,
	%every axis plot/.append style={line width=1.0pt}
	legend pos=north east,
	]
	
	\addlegendentry{Eigener Datensatz}; % legende1
	\addplot[gray, line width=0.8pt]  table [x=Step,y=Value,col sep=comma] from {Bilder/hog.csv};
	\addplot[black, line width=0.8pt]  table [x=Step,y=Value,col sep=comma] from {Bilder/hogcoco.csv};
	\addlegendimage{/pgfplots/refstyle=plotbrr};				
	\addlegendentry{\textit{COCO}-Datensatz} ;%legende 2;
	%\addlegendentry{plot 1}
	\end{axis}
	\end{tikzpicture}
	\caption{\textit{Precision-Recall}-Kurven des \textit{SSD MobileNet V2} Netz. Die schwarze Kurve beschreibt die Analyse durch den \textit{COCO}-Datensatz. Der Test mithilfe des eigenen Datensatz wir durch die graue Kurve präsentiert. https://github.com/kaka-lin/object-detection}
	\label{fig: ssdmobilenetv2}
\end{figure}


Der Kombination aus \textit{HoG} und \textit{SVM} erreicht trotz der höheren Auflösung der eingehenden Bilder geringe Werte laut den in Abbildung \ref{fig: hog1} präsentierten Ergebnisse. Der \textit{Recall}-Wert liegt für den Test mit dem eigenen Datensatz konstant unter 0,35 bei einem maximalen \textit{Precision}-Wert von knapp unter 0,55. Bei der Durchsicht der eingetragenen Begrenzungsrahmen ist aufgefallen, dass diese verhältnismäßig groß ausfallen. Somit könnte der \textit{IoU}-Wert entsprechend niedrig sein un zu diesem Ergebnis führen.



\begin{figure}[H]
	\centering
	\begin{tikzpicture}[
	]
	\begin{axis}[
	width=12cm,
	height=7cm,
	axis y line*=left,
	ticklabel style={% gilt für x und y
		/pgf/number format/.cd,
		use comma,% Komma als Dezimaltrenner
		1000 sep = {}% keine Tausendertrennung 
	},
	xlabel={$\text{Recall } \textit{r(t)}$},
	ylabel={$\text{Precision } \textit{p(t)}$},
	axis x line*=bottom,
	xmin=0, xmax=1, 
	ymin=0, ymax=1.2,
	%every axis plot/.append style={line width=1.0pt}
	legend pos=north east,
	]
	
	\addlegendentry{Eigener Datensatz}; % legende1
	\addplot[gray, line width=0.8pt]  table [x=Step,y=Value,col sep=comma] from {Bilder/cocossdmobilenetv1tflite.csv};
	\addplot[black, line width=0.8pt]  table [x=Step,y=Value,col sep=comma] from {Bilder/cocossdmobilenetv1cocotest.csv};
	\addlegendimage{/pgfplots/refstyle=plotbrr};				
	\addlegendentry{\textit{COCO}-Datensatz} ;%legende 2;
	%\addlegendentry{plot 1}
	\end{axis}
	\end{tikzpicture}
	\caption{Gegenüberstellung der \textit{Precision} und \textit{Recall} Kurven eines quantisierten \textit{Tensorflow Lite SSD MobileNet V1} Netzes. Durch die schwarze Kurve wird das Testergebnis durch den \textit{COCO}-Datensatz gezeigt. Die graue Kurve zeigt das \textit{Precision} und \textit{Recall} Verhältnis für die Analyse des eigenen Datensatzes. Tensorflow starter}
	\label{fig: ssdmobilenetv1}
\end{figure}


Eine deutliche Steigerung hinsichtlich der Geschwindigkeit wird durch das \textit{Tensorflow Lite SSD MobileNet V1} Modell erreicht. Eine präzise Auslistung aller gemessenen Analysezeiten ist in Tabelle \ref{fig: zeitentab} präsentiert. Die Genauigkeit scheint bei diesem Netz zunächst niedriger auszufallen als erwartet. Jedoch spielt die Komplexität des Datensatzes eine große Rolle. Im Paper von \textit{Huang} werden die Genauigkeitswerte verschiedener KNNs gegenübergestellt \cite{maxssdmobilenet}. Die dort präsentierte, mittlere Durchschnittsgenauigkeit (\textit{Overall mAP}) basierend auf den \textit{COCO}-Datensatz liegt bei circa 0,2 \cite{maxssdmobilenet}. 


\begin{figure}[H]
	\centering
	\begin{tikzpicture}[
	]
	\begin{axis}[
	width=12cm,
	height=7cm,
	axis y line*=left,
	ticklabel style={% gilt für x und y
		/pgf/number format/.cd,
		use comma,% Komma als Dezimaltrenner
		1000 sep = {}% keine Tausendertrennung 
	},
	xlabel={$\text{Recall } \textit{r(t)}$},
	ylabel={$\text{Precision } \textit{p(t)}$},
	axis x line*=bottom,
	xmin=0, xmax=1, 
	ymin=0, ymax=1.2,
	%every axis plot/.append style={line width=1.0pt}
	legend pos=north east,
	legend cell align={left},
	]
	
	\addlegendentry{\footnotesize Eigener Datensatz}; % legende1
	\addplot[gray, line width=0.8pt]  table [x=Step,y=Value,col sep=comma] from {Bilder/ownssdmobilenetv1.csv};
	\addplot[line width=0.8pt]  table [x=Step,y=Value,col sep=comma] from {Bilder/ownnetv1coco.csv};
	\addlegendimage{/pgfplots/refstyle=plotbrr};				
	\addlegendentry{\footnotesize \textit{COCO}-Datensatz} ;%legende 2;
	%\addlegendentry{plot 1}
	\end{axis}
	\end{tikzpicture}
	\caption{\textit{Precision-Recall}-Kurven des entwickelten \textit{MobileNet V1 SSD} Netzes.}
	\label{fig: ownnetv1}
\end{figure}


Das entwickelte \textit{SSD-MobileNet V1}...Die Trainingskonfigurationen dieses sowie des entwickelten \textit{SSD MobileNet V2} Netzes sind in Anhang ... gezeigt.


\begin{figure}[H]
	\centering
	\begin{tikzpicture}[
	]
	\begin{axis}[
	width=12cm,
	height=7cm,
	axis y line*=left,
	ticklabel style={% gilt für x und y
		/pgf/number format/.cd,
		use comma,% Komma als Dezimaltrenner
		1000 sep = {}% keine Tausendertrennung 
	},
	xlabel={$\text{Recall } \textit{r(t)}$},
	ylabel={$\text{Precision } \textit{p(t)}$},
	axis x line*=bottom,
	xmin=0, xmax=1, 
	ymin=0, ymax=1.2,
	%every axis plot/.append style={line width=1.0pt}
	legend pos=north east,
	]
	
	\addlegendentry{Eigener Datensatz}; % legende1
	\addplot[gray, line width=0.8pt]  table [x=Step,y=Value,col sep=comma] from {Bilder/ssdmobilenetv2tflite.csv};
	\addplot[black, line width=0.8pt]  table [x=Step,y=Value,col sep=comma] from {Bilder/ssdmobilenetv2tflitecoco.csv};
	\addlegendimage{/pgfplots/refstyle=plotbrr};				
	\addlegendentry{\textit{COCO}-Datensatz} ;%legende 2;
	%\addlegendentry{plot 1}
	\end{axis}
	\end{tikzpicture}
	\caption{\textit{Precision-Recall}-Kurven eines \textit{SSD MobileNet V2} Netz \cite{ssdv2}. }
	\label{fig: ssdmobilenetv2}
\end{figure}


Abbildung \ref{fig: ssdmobilenetv2} zeigt die \textit{Precision} und \textit{Recall} Kurve des \textit{Tensorflow Lite} Modells mit der \textit{SSD MobileNet V2} Architektur. Auch in diesem Fall erreicht das Netz erwartungsgemäß bei der Analyse durch den \textit{COCO}-Datensatz geringere Werte als durch den eigenen Datensatz. 


\begin{figure}[H]
	\centering
	\begin{tikzpicture}[
	]
	\begin{axis}[
	width=12cm,
	height=7cm,
	axis y line*=left,
	ticklabel style={% gilt für x und y
		/pgf/number format/.cd,
		use comma,% Komma als Dezimaltrenner
		1000 sep = {}% keine Tausendertrennung 
	},
	xlabel={$\text{Recall } \textit{r(t)}$},
	ylabel={$\text{Precision } \textit{p(t)}$},
	axis x line*=bottom,
	xmin=0, xmax=1, 
	ymin=0, ymax=1.2,
	%every axis plot/.append style={line width=1.0pt}
	legend pos=north east,
	]
	
	\addlegendentry{Eigener Datensatz}; % legende1
	\addplot[gray, line width=0.8pt]  table [x=Step,y=Value,col sep=comma] from {Bilder/ownnetv2.csv};
	\addplot[line width=0.8pt]  table [x=Step,y=Value,col sep=comma] from {Bilder/ownnetv2coco.csv};
	
	\addlegendimage{/pgfplots/refstyle=plotbrr};				
	\addlegendentry{\textit{COCO}-Datensatz} ;%legende 2;
	%\addlegendentry{plot 1}
	\end{axis}
	\end{tikzpicture}
	\caption{\textit{Precision-Recall}-Kurven des enwtickelten \textit{SSD MobileNet V2} Netz.}
	\label{fig: ssdmobilenetv2}
\end{figure}
   


\begin{figure}[H]
	\centering
	\begin{tikzpicture}[
	]
	\begin{axis}[
	width=12cm,
	height=7cm,
	axis y line*=left,
	ticklabel style={% gilt für x und y
		/pgf/number format/.cd,
		use comma,% Komma als Dezimaltrenner
		1000 sep = {}% keine Tausendertrennung 
	},
	xlabel={$\text{Recall } \textit{r(t)}$},
	ylabel={$\text{Precision } \textit{p(t)}$},
	axis x line*=bottom,
	xmin=0.5, xmax=0.9, 
	ymin=0, ymax=1.2,
	%every axis plot/.append style={line width=1.0pt}
	%legend pos=north east,
	legend cell align={left},
	legend style={at={(axis cs:0.75,1)},anchor=south west} ,
	]
	
	\addlegendentry{\footnotesize V2 SSDLite (90)}; % legende1
	\addplot[gray!60, line width=0.8pt]  table [x=Step,y=Value,col sep=comma] from {Bilder/ssdmobilenetv2tflite.csv};
%	\addplot[line width=0.8pt]  table [x=Step,y=Value,col sep=comma] from {Bilder/hog.csv};
	\addplot[dashed,line width=0.8pt]  table [x=Step,y=Value,col sep=comma] from {Bilder/ownnetv2.csv};
	\addplot[gray!60,dashed,line width=0.8pt]  table [x=Step,y=Value,col sep=comma] from {Bilder/ownnetv2ssdlite.csv};
	\addplot[dotted,line width=0.8pt]  table [x=Step,y=Value,col sep=comma] from {Bilder/ownssdmobilenetv1.csv};
	\addplot[dash dot,line width=0.8pt]  table [x=Step,y=Value,col sep=comma] from {Bilder/cocossdmobilenetv1tflite.csv};
	\addlegendimage{/pgfplots/refstyle=plotbrr};				
%	\addlegendentry{HoG \& SVM (H)} ;
	\addlegendentry{\footnotesize mod. V2 SSD (1)} ;%legende 2;
	\addlegendentry{\footnotesize mod. V2 SSDLite (1)} ;%legende 2;
	\addlegendentry{\footnotesize mod. V1 SSD (1)} ;%legende 2;
	\addlegendentry{\footnotesize quant. V1 SSD (90)} ;%legende 2;
	%\addlegendentry{plot 1}
	\end{axis}
	\end{tikzpicture}
	\caption{\textit{Precision-Recall}-Kurven aller Objekterkennungssysteme in Anwendung auf den eigenen Datensatz. Das Kürzel \textit{H} steht für heruntergeladene Systeme und das \textit{E} für entwickelte.}
	\label{fig: genauigkeitsvergleich}

	
\end{figure}


Im Gesamtvergleich der Genauigkeit in Anwendung auf den eigenen Datensatz aller Untersuchten Systeme sticht die ... Architektur als das Verfahren mit den höchsten \textit{Precision-Recall} Werten heraus. 

\begin{table}[H]
	\caption{Vergleich der Rechenzeiten pro Bild auf verschiedenen Hardwareplattformen. Die präsentierten Zeiten wurden für alle Analyseschritte addiert und durch die Anzahl aller Bilder geteilt. Ein Analyseschritt bedeutet in diesem Fall die reine Berechnung des Netzes und exkludiert beispielsweise die Zeit für eine Anpassung des Bildes für das entsprechende Netz.  }
	\begin{center}
		\begin{tabular}{|c|c|c|c|c|}
			\hline
			\multicolumn{1}{|c|}{Hardware} & \multicolumn{1}{c|}{Hog \& SVM} & \multicolumn{1}{c|}{SSD MobileNet V1} & \multicolumn{1}{c|}{SSD MobileNet V2} \\ \hline
			Computer ALF	&73.2 	&69 ms		& 600 $\times$ 1000 	 \\
			Grafikkarte			&77.2 	&46		& 	300 $\times$ 300  	 \\
			Raspberry Pi Model B+			&68.0	&59		& 300 $\times$ 300 \\
			
			\hline
		\end{tabular}
	\end{center}

	\label{fig: mobilessdtab}
\end{table}

Die Berechnungszeit pro Bild des eigenen Datensatzes aller untersuchten Objekterkennungssysteme ist in der Tabelle in Abbildung \ref{fig: zeitenvergleich} präsentiert. \\

Die Auswahl der entsprechenden Systeme ist nach wie vor abhängig vond er verwendeten Hardware. Das Ziel in diesem Projekt ist der Einsatz am integrierten Computer des ALFs sowie eine mögliche AUslagerung auf ein eingebettetes System. 



