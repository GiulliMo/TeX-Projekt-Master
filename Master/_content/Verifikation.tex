\chapter{Evaluation}
\label{ch: Verifikation}
Zur Evaluation der künstlichen neuronalen Netze wird eine anwendungsorientierte \textit{Benchmark} durchgeführt. Hierbei wird anhand der in Kapitel \ref{subsec: Entwickeltes neuronales Netz} beschriebenen Datensätze die \textit{Precision-Recall}-Methode angewendet. Weiterhin werden die Benchmarks auf dem integrierten Computer des ALFs und einem eingebettetem System ausgeführt. Als eingebettetes System wird ein \textit{Raspberry Pi 3 Model B} verwendet. Die Eckdaten des integrierten Computers sind in der Masterthesis von M.Sc. \textit{Dominik Eickmann} und M.Sc. \textit{Dennis Hotze} dargestellt \cite{alf}. Es ist keine geräteübergreifende Veränderung der Genauigkeit je Netz zu erwarten. Jedoch können so die Bearbeitungszeiten pro Bild für unterschiedliche Hardware verglichen werden. Eine präzise Auflistung aller gemessenen Analysezeiten ist in Tabelle \ref{fig: zeitentab} präsentiert. \\

Insgesamt werden für den Test durch \textit{COCO-Datensatz} 12755 Bilder aus dem Trainingsdatensatz verwendet. Der technische Hintergrund hierfür ist in Kapitel \ref{subsec: Entwickeltes neuronales Netz} zu finden. Als Vergleich analysiert jedes Netz auch den eigenen Datensatz. So kann die Performance am Einsatzort des ALFs an der Hochschule Bochum evaluiert werden. Jedes Bild wird für die verwendeten, neuronalen Netze auf eine Pixelgröße von $300 \times 300$ skaliert. Für die Evaluation des \textit{HoGs} wird eine höhere Auflösung gewählt. Hierbei wird eine Seite des Bildes softwareseitig auf 400 Pixel begrenzt. \\

Der in Grundlagenkapitel \ref{subsec: evaluation neuronaler netze} beschriebene mAP-Wert wird häufig auf Objekterkennungssystemen mit multiplen Klassen angewendet. Die hier entwickelte Personenerkennung soll jedoch lediglich die Klasse \textit{Person} erkennen. Somit ist der mAP-Wert in diesem Fall der Mittelwert eines Messwerts und kann als Integral der \textit{Precision-Recall}-Kurve angesehen werden. Im Verlauf der Evaluation der angewendeten Systeme wird mithilfe einer Berechnungssoftware jeweils der mAP-Wert berechnet.
 

\begin{figure}[H]
	\centering
	\begin{tikzpicture}[
	]
	\begin{axis}[
	width=12cm,
	height=7cm,
	axis y line*=left,
	ticklabel style={% gilt für x und y
		/pgf/number format/.cd,
		use comma,% Komma als Dezimaltrenner
		1000 sep = {}% keine Tausendertrennung 
	},
	xlabel={$\text{Recall } \textit{r(t)}$},
	ylabel={$\text{Precision } \textit{p(t)}$},
	axis x line*=bottom,
	xmin=0, xmax=0.3, 
	ymin=0, ymax=0.3,
	%every axis plot/.append style={line width=1.0pt}
	legend pos=north east,
	]
	
	\addlegendentry{Eigener Datensatz}; % legende1
	\addplot[gray, line width=0.8pt]  table [x=Step,y=Value,col sep=comma] from {Bilder/hog.csv};
	\addplot[black, line width=0.8pt]  table [x=Step,y=Value,col sep=comma] from {Bilder/hogcoco.csv};
	\addlegendimage{/pgfplots/refstyle=plotbrr};				
	\addlegendentry{\textit{COCO}-Datensatz} ;%legende 2;
	%\addlegendentry{plot 1}
	\end{axis}
	\end{tikzpicture}
	\caption{\textit{Precision-Recall}-Kurven des \textit{SSD MobileNet V2} Netz. Die schwarze Kurve beschreibt die Analyse durch den \textit{COCO}-Datensatz. Der Test mithilfe des eigenen Datensatz wir durch die graue Kurve präsentiert. https://github.com/kaka-lin/object-detection}
	\label{fig: ssdmobilenetv2}
\end{figure}


Die Kombination aus \textit{HoG} und \textit{SVM} erreicht in der Benchmark die in Abbildung \ref{fig: hog1} präsentierten Ergebnisse. Der \textit{mAP}-Wert liegt für den eigenen Datensatz bei 0,16 und für den \textit{COCO}-Datensatz bei 0,07 Bei der Durchsicht der eingetragenen Begrenzungsrahmen ist aufgefallen, dass diese verhältnismäßig groß ausfallen. Somit könnte der \textit{IoU}-Wert entsprechend niedrig sein und zu diesem Ergebnis führen. Diese Beobachtung kann der \textit{HoG-SVM}-Methode zugrunde liegen. Mit den entsprechende Optimierungen wurden die \textit{Precision-Recall} Werte für dieses System maximiert und sind in der obigen Abbildung dargestellt.



\begin{figure}[H]
	\centering
	\begin{tikzpicture}[
	]
	\begin{axis}[
	width=12cm,
	height=7cm,
	axis y line*=left,
	ticklabel style={% gilt für x und y
		/pgf/number format/.cd,
		use comma,% Komma als Dezimaltrenner
		1000 sep = {}% keine Tausendertrennung 
	},
	xlabel={$\text{Recall } \textit{r(t)}$},
	ylabel={$\text{Precision } \textit{p(t)}$},
	axis x line*=bottom,
	xmin=0, xmax=1, 
	ymin=0, ymax=1.2,
	%every axis plot/.append style={line width=1.0pt}
	legend pos=north east,
	]
	
	\addlegendentry{Eigener Datensatz}; % legende1
	\addplot[gray, line width=0.8pt]  table [x=Step,y=Value,col sep=comma] from {Bilder/cocossdmobilenetv1tflite.csv};
	\addplot[black, line width=0.8pt]  table [x=Step,y=Value,col sep=comma] from {Bilder/cocossdmobilenetv1cocotest.csv};
	\addlegendimage{/pgfplots/refstyle=plotbrr};				
	\addlegendentry{\textit{COCO}-Datensatz} ;%legende 2;
	%\addlegendentry{plot 1}
	\end{axis}
	\end{tikzpicture}
	\caption{Gegenüberstellung der \textit{Precision} und \textit{Recall} Kurven eines quantisierten \textit{Tensorflow Lite SSD MobileNet V1} Netzes. Durch die schwarze Kurve wird das Testergebnis durch den \textit{COCO}-Datensatz gezeigt. Die graue Kurve zeigt das \textit{Precision} und \textit{Recall} Verhältnis für die Analyse des eigenen Datensatzes. Tensorflow starter}
	\label{fig: ssdmobilenetv1}
\end{figure}


Eine deutliche Steigerung hinsichtlich der Geschwindigkeit im Vergleich zur \textit{HoG-SVM}-Methode wird durch das quantisierte \textit{MobileNet V1 SSD} Netz erreicht. Die Berechnung des Integrals der \textit{Precision-Recall}-Kurve ergab für den eigenen Datensatz einen Wert von 0,68 und für den \textit{COCO}-Datensatz 0,46. Dieses Netz ist in der Lage 90 verschiedene Klassen zu erkennen. Hierbei nimmt das Netz jedoch lediglich 4 MB Speicherplatz ein.


\begin{figure}[H]
	\centering
	\begin{tikzpicture}[
	]
	\begin{axis}[
	width=12cm,
	height=7cm,
	axis y line*=left,
	ticklabel style={% gilt für x und y
		/pgf/number format/.cd,
		use comma,% Komma als Dezimaltrenner
		1000 sep = {}% keine Tausendertrennung 
	},
	xlabel={$\text{Recall } \textit{r(t)}$},
	ylabel={$\text{Precision } \textit{p(t)}$},
	axis x line*=bottom,
	xmin=0, xmax=1, 
	ymin=0, ymax=1.2,
	%every axis plot/.append style={line width=1.0pt}
	legend pos=north east,
	legend cell align={left},
	]
	
	\addlegendentry{\scriptsize Eigener Datensatz}; % legende1
	\addplot[gray!50, line width=1pt]  table [x=Step,y=Value,col sep=comma] from {Bilder/ownssdmobilenetv1.csv};
	\addplot[line width=1pt]  table [x=Step,y=Value,col sep=comma] from {Bilder/ownnetv1coco.csv};
	\addlegendimage{/pgfplots/refstyle=plotbrr};				
	\addlegendentry{\scriptsize \textit{COCO}-Datensatz} ;%legende 2;
	%\addlegendentry{plot 1}
	\end{axis}
	\end{tikzpicture}
	\caption{\textit{Precision-Recall} Kurven des entwickelten \textit{MobileNet V1 SSD} Netzes.}
	\label{fig: ownnetv1}
\end{figure}


Das in Abbildung \ref{fig: ownnetv1} gezeigte \textit{MobileNet V1 SSD}-Netz ist darauf trainiert Personen zu erkennen. Die dargestellte Architektur weißt einen \textit{mAP}-Werte von 0,79 für den eigenen Datzensatz und 0,56 für den \textit{COCO}-Datensatz auf.


\begin{figure}[H]
	\centering
	\begin{tikzpicture}[
	]
	\begin{axis}[
	width=12cm,
	height=7cm,
	axis y line*=left,
	ticklabel style={% gilt für x und y
		/pgf/number format/.cd,
		use comma,% Komma als Dezimaltrenner
		1000 sep = {}% keine Tausendertrennung 
	},
	xlabel={$\text{Recall } \textit{r(t)}$},
	ylabel={$\text{Precision } \textit{p(t)}$},
	axis x line*=bottom,
	xmin=0, xmax=1, 
	ymin=0, ymax=1.2,
	%every axis plot/.append style={line width=1.0pt}
	legend cell align={left},
	legend pos=north east,
	]
	
	\addlegendentry{\scriptsize Eigener Datensatz}; % legende1
	\addplot[gray, line width=0.8pt]  table [x=Step,y=Value,col sep=comma] from {Bilder/ownnetv2.csv};
	\addplot[line width=0.8pt]  table [x=Step,y=Value,col sep=comma] from {Bilder/ownnetv2coco.csv};
	
	\addlegendimage{/pgfplots/refstyle=plotbrr};				
	\addlegendentry{\scriptsize \textit{COCO}-Datensatz} ;%legende 2;
	%\addlegendentry{plot 1}
	\end{axis}
	\end{tikzpicture}
	\caption{\textit{Precision-Recall}-Kurven des enwtickelten \textit{SSD MobileNet V2} Netz.}
	\label{fig: ownnetv2}
\end{figure}


In Darstellung \ref{fig: ownnetv2} werden die \textit{Precision-Recall}-Kurven eines \textit{MobileNet V2 SSD} gezeigt. Das Verhaltensmuster dieses Netzes ist ebenfalls auf eine Personenerkennung beschränkt, um enthaltene Parameter und den damit verbundenen Speicherplatz zu reduzieren. Die \textit{mAP}-Werte liegen bei 0,78 für den eigenen Datensatz und 0,54 für den \textit{COCO}-Datensatz. 


\begin{figure}[H]
	\centering
	\begin{tikzpicture}[
	]
	\begin{axis}[
	width=12cm,
	height=7cm,
	axis y line*=left,
	ticklabel style={% gilt für x und y
		/pgf/number format/.cd,
		use comma,% Komma als Dezimaltrenner
		1000 sep = {}% keine Tausendertrennung 
	},
	xlabel={$\text{Zeit } \textit{t} \text{ in Sekunden}$},
	ylabel={Drehmoment $M_{soll}$ in \%M$_{\mathrm{max}}$},
	axis x line*=bottom,
	xmin=0, xmax=1, 
	ymin=0, ymax=1.2,
	%every axis plot/.append style={line width=1.0pt}
	legend pos=north east,
	]
	
	\addlegendentry{Soll-Drehmoment}; % legende1
	\addplot[gray, line width=0.8pt]  table [x=Step,y=Value,col sep=comma] from {Bilder/ssdmobilenetv2tflite.csv};
	\addlegendimage{/pgfplots/refstyle=plotbrr};				
	\addlegendentry{Ist-Drehzahl} ;%legende 2;
	%\addlegendentry{plot 1}
	\end{axis}
	\end{tikzpicture}
	\caption{https://github.com/kaka-lin/object-detection}
	\label{fig: brr}
\end{figure}


Abbildung \ref{fig: ssdmobilenetv2} zeigt die \textit{Precision} und \textit{Recall} Kurve einer \textit{MobileNet V2 SSDLite} Architektur. Auffällig hierbei, ist die konstant hohe Genauigkeit $p(t)$. Diese liegt bis zu einem \textit{Recall}-Wert von circa 0,8 zwischen 0,9 und 1. Für die Anwendung auf den eigenen Datensatz erreicht diese Architektur eine mittlere Durschnittsgenauigkeit von 0,77. Angewendet auf den \textit{COCO}-Datensatz liegt der Wert bei 0,54. 

 

\begin{figure}[H]
	\centering
	\begin{tikzpicture}[
	]
	\begin{axis}[
	width=12cm,
	height=7cm,
	axis y line*=left,
	ticklabel style={% gilt für x und y
		/pgf/number format/.cd,
		use comma,% Komma als Dezimaltrenner
		1000 sep = {}% keine Tausendertrennung 
	},
	xlabel={$\text{Recall } \textit{r(t)}$},
	ylabel={$\text{Precision } \textit{p(t)}$},
	axis x line*=bottom,
	xmin=0, xmax=1, 
	ymin=0, ymax=1.2,
	%every axis plot/.append style={line width=1.0pt}
	legend pos=north east,
	legend cell align={left},
	]
	
	\addlegendentry{Eigener Datensatz}; % legende1
	\addplot[gray, line width=0.8pt]  table [x=Step,y=Value,col sep=comma] from {Bilder/ownnetv2ssdlite.csv};
	\addplot[line width=0.8pt]  table [x=Step,y=Value,col sep=comma] from {Bilder/ownnetv2ssdlitecoco.csv};
	
	\addlegendimage{/pgfplots/refstyle=plotbrr};				
	\addlegendentry{\textit{COCO}-Datensatz} ;%legende 2;
	%\addlegendentry{plot 1}
	\end{axis}
	\end{tikzpicture}
	\caption{Darstellung der \textit{Precision-Recall}-Verläufe des enwtickelten \textit{MobileNet V2 SSDLite} Netz.}
	\label{fig: ownnetv2ssdlite}
\end{figure}

 
  
Die \textit{Precision-Recall}-Kurven der \textit{MobileNet V2 SSDLite} Architektur sind in der Grafik \ref{fig: ownnetv2ssdlite} abgebildet. \\

Nachfolgend werden die Benchmarkergebnisse der untersuchten Objekterkennungssysteme gegenübergestellt. Der Vergleich dient zur Veranschaulichung der gemessenen \textit{Precision-Recall}-Leistung je Datensatz. In Abbildung \ref{fig: genauigkeitsvergleich} werden die untersuchten \textit{MobileNet}-Architekturen anhand des eigenen Datensatzes verglichen. So kann eine Aussage darüber getroffen werden, ob die Systeme ortsabhängig ein anderes Verhaltensmuster an aufzeigen. Es kann beispielsweise passieren, dass die Netze aufgrund prägnater Eigenschaften der Umgebung am Standort der Hochschule Bochum verschieden reagieren. Eine allgemeine Aussage über die Genauigkeiten kann anhand der Darstellung \ref{fig: genauigkeitsvergleichcoco} getroffen werden. Für die Verallgemeinerung sorgt hierbei der \textit{COCO}-Datensatz. Dieser enthält anders als der eigene Datensatz Bilder von verschiedenen Orten.     



\begin{figure}[H]
	\centering
	\begin{tikzpicture}[
	]
	\begin{axis}[
	width=12cm,
	height=7cm,
	axis y line*=left,
	ticklabel style={% gilt für x und y
		/pgf/number format/.cd,
		use comma,% Komma als Dezimaltrenner
		1000 sep = {}% keine Tausendertrennung 
	},
	xlabel={$\text{Recall } \textit{r(t)}$},
	ylabel={$\text{Precision } \textit{p(t)}$},
	axis x line*=bottom,
	xmin=0.0, xmax=1.0, 
	ymin=0, ymax=1.2,
	%every axis plot/.append style={line width=1.0pt}
	%legend pos=north east,
	legend cell align={left},
	legend style={at={(0.5,1)},anchor=north} ,
	legend columns = 3,
	]
	
	\addlegendentry{\scriptsize V2 SSDLite (90)}; % legende1
	\addplot[gray!50, line width=1pt]  table [x=Step,y=Value,col sep=comma] from {Bilder/ssdmobilenetv2tflite.csv};
%	\addplot[line width=0.8pt]  table [x=Step,y=Value,col sep=comma] from {Bilder/hog.csv};
	\addplot[dashed,line width=1pt]  table [x=Step,y=Value,col sep=comma] from {Bilder/ownnetv2.csv};
	\addplot[gray!50,dashed,line width=1pt]  table [x=Step,y=Value,col sep=comma] from {Bilder/ownnetv2ssdlite.csv};
	\addplot[dotted,line width=1pt]  table [x=Step,y=Value,col sep=comma] from {Bilder/ownssdmobilenetv1.csv};
	\addplot[dash dot,line width=1pt]  table [x=Step,y=Value,col sep=comma] from {Bilder/cocossdmobilenetv1tflite.csv};
	\addplot[line width=1pt]  table [x=Step,y=Value,col sep=comma] from {Bilder/hog.csv};
	\addlegendimage{/pgfplots/refstyle=plotbrr};				
%	\addlegendentry{HoG \& SVM (H)} ;
	\addlegendentry{\scriptsize mod. V2 SSD (1)} ;%legende 2;
	\addlegendentry{\scriptsize mod. V2 SSDLite (1)} ;%legende 2;
	\addlegendentry{\scriptsize mod. V1 SSD (1)} ;%legende 2;
	\addlegendentry{\scriptsize quant. V1 SSD (90)} ;%legende 2;
	\addlegendentry{\scriptsize HOG \& SVM} ;%legende 2;
	%\addlegendentry{plot 1}
	\end{axis}
	\end{tikzpicture}
	\caption{\textit{Precision-Recall} Kurven aller \textit{MobileNet} Architekturen in Anwendung auf den eigenen Datensatz. Die Abkürzungen \textit{mod.} und \textit{quant.} stehen für \textit{modifiziert} und \textit{quantisiert}.}
	\label{fig: genauigkeitsvergleich}

	
\end{figure}



\begin{figure}[H]
	\centering
		\begin{tikzpicture}[
	]
	\begin{axis}[
		width=12cm,
		height=7cm,
		axis y line*=left,
		ticklabel style={% gilt für x und y
			/pgf/number format/.cd,
			use comma,% Komma als Dezimaltrenner
			1000 sep = {}% keine Tausendertrennung 
		},
		xlabel={$\text{Recall } \textit{r(t)}$},
		ylabel={$\text{Precision } \textit{p(t)}$},
		axis x line*=bottom,
		xmin=0.0, xmax=0.7, 
		ymin=0, ymax=1.2,
		%every axis plot/.append style={line width=1.0pt}
		%legend pos=north east,
		legend cell align={left},
		legend style={at={(axis cs:0.5,1)},anchor=south west} ,
		]
		
		\addlegendentry{\footnotesize V2 SSDLite (90)}; % legende1
		\addplot[gray!60, line width=0.8pt]  table [x=Step,y=Value,col sep=comma] from {Bilder/ssdmobilenetv2tflitecoco.csv};
		%	\addplot[line width=0.8pt]  table [x=Step,y=Value,col sep=comma] from {Bilder/hog.csv};
		\addplot[dashed,line width=0.8pt]  table [x=Step,y=Value,col sep=comma] from {Bilder/ownnetv2coco.csv};
		\addplot[gray!60,dashed,line width=0.8pt]  table [x=Step,y=Value,col sep=comma] from {Bilder/ownnetv2ssdlitecoco.csv};
		\addplot[dotted,line width=0.8pt]  table [x=Step,y=Value,col sep=comma] from {Bilder/ownnetv1coco.csv};
		\addplot[dash dot,line width=0.8pt]  table [x=Step,y=Value,col sep=comma] from {Bilder/cocossdmobilenetv1cocotest.csv};
		\addlegendimage{/pgfplots/refstyle=plotbrr};				
		%	\addlegendentry{HoG \& SVM (H)} ;
		\addlegendentry{\footnotesize mod. V2 SSD (1)} ;%legende 2;
		\addlegendentry{\footnotesize mod. V2 SSDLite (1)} ;%legende 2;
		\addlegendentry{\footnotesize mod. V1 SSD (1)} ;%legende 2;
		\addlegendentry{\footnotesize quant. V1 SSD (90)} ;%legende 2;
		%\addlegendentry{plot 1}
	\end{axis}
\end{tikzpicture}
	\caption{\textit{Precision-Recall}-Kurven aller Objekterkennungssysteme in Anwendung auf den eigenen Datensatz. Das Kürzel \textit{H} steht für heruntergeladene Systeme und das \textit{E} für entwickelte.}
	\label{fig: genauigkeitsvergleichcoco}
	
	

	
\end{figure}


Bei den Auswertungen der \textit{Precision-Recall}-Werte erreichte die Kombination aus \textit{HoG} und \textit{SVM} im Vergleich zu den \textit{MobileNet}-Netzen niedrige Ergebnisse. In den Vergleichsgrafiken \ref{fig: genauigkeitsvergleich} und \ref{fig: genauigkeitsvergleichcoco} wird das System aufgrunddessen nicht weiter betrachtet. Es ist zu beachten, dass die verwendete Auswertungssoftware den \textit{Recall} in einer Schrittweite von 0,25 ausgibt. Überwiegend fällt das quantisierte \textit{MobileNet V1 SSD}-Netz in Abbildung \ref{fig: genauigkeitsvergleich} auf. Die Kurve zeigt einen anderen Verlauf als die anderen Kandidaten. Das Verhältnis aus \textit{Precision} und \textit{Recall} reduziert sich bei derartigen Strukturen durch eine Quantisierung. Netzoptimierungen wie die zweite Version des \textit{MobileNets} oder der Weiterentwicklung des \textit{SSDs} zeigen vorwiegend eine Beschränkung der \textit{Recall}-Werte. Jedoch gibt es im Verlauf der \textit{Precision}-Werte keine nenneswerte Unterschiede. 

\begin{table}[H]
	\caption{Vergleich der Rechenzeiten pro Bild auf verschiedenen Hardwareplattformen. Die präsentierten Zeiten wurden für alle Analyseschritte addiert und durch die Anzahl aller Bilder geteilt. Ein Analyseschritt bedeutet in diesem Fall die reine Berechnung des Netzes und exkludiert beispielsweise die Zeit für eine Anpassung des Bildes für das entsprechende Netz.  }
	\begin{center}
		\begin{tabular}{|c|c|c|c|c|}
			\hline
			\multicolumn{1}{|c|}{Hardware} & \multicolumn{1}{c|}{Hog \& SVM} & \multicolumn{1}{c|}{SSD MobileNet V1} & \multicolumn{1}{c|}{SSD MobileNet V2} \\ \hline
			Computer ALF	&- 	&69 ms		& - 	 \\
			Grafikkarte			&-	&-		& 	-  	 \\
			Raspberry Pi 3 Model B+			&-	&-		&- \\
			
			\hline
		\end{tabular}
	\end{center}

	\label{fig: zeitentab}
\end{table}

Die Berechnungszeit pro Bild des eigenen Datensatzes aller untersuchten Objekterkennungssysteme ist in der Tabelle in Abbildung \ref{fig: zeitentab} präsentiert. 

\begin{table}[H]
	\caption{Vergleich der Rechenzeiten pro Bild auf verschiedenen Hardwareplattformen. Die präsentierten Zeiten wurden für alle Analyseschritte addiert und durch die Anzahl aller Bilder geteilt. Ein Analyseschritt bedeutet in diesem Fall die reine Berechnung des Netzes und exkludiert beispielsweise die Zeit für eine Anpassung des Bildes für das entsprechende Netz.  }
	\begin{center}
		\begin{tabular}{|c|c|c|c|c|}
			\hline
			\multicolumn{1}{|c|}{} & \multicolumn{1}{c|}{HoG \& SVM} & \multicolumn{1}{c|}{SSD MobileNet quant V1 H/E} & \multicolumn{1}{c|}{SSD MobileNet V2 H/E} \\ \hline 
			Eigener Datensatz	&0,16 	&0,61 / 0,57		& 0,77 / 0,47	 \\
			\textit{COCO} Datensatz	& 	&0,5 / 0,42		&  0,5 / 0,34	 \\
			
			\hline
		\end{tabular}
	\end{center}

	\label{fig: genauigkeiten}
\end{table}

