\chapter{Zusammenfassung und Ausblick}
\label{Fazit und Ausblick}
	
Im Rahmen dieser Masterarbeit wurde eine Personenerkennung für die Anwendung am autonomen Logistikfahrzeug ALF entwickelt und evaluiert. Die entwickelte Software ist darauf optimiert, eine ausgewogene Balance zwischen einer hohen Genauigkeit bei kurzer Rechenzeit zu erzielen. Dies wird durch eine Unterteilung des Gesamtsystems in Teilsysteme auf verschiedenen Ebenen erreicht. Die implementierten Softwarekomponenten eignen sich für den Einsatz auf einem eingebetteten Systemen zur Entlastung des ALFS hinsichtlich des Rechenaufwands und wurden bewusst danach ausgelegt.\\

Eine dem Erkennungssystem übergeordnete Personendetektion erkennt Menschen im unmittelbaren Sichtkegel der verbauten \textit{Kinect}-Kameras. Die Bildverarbeitung nutzt ein künstliches neuronales Netzwerk zur Analyse von Bildern. Für die Auswahl der Netze wurden state-of-the-art Lösungen verglichen und mögliche Netztypen evaluiert.\\

Im Hinblick auf die in Kapitel \ref{ch: Verifikation} erarbeiteten Genauigkeitswerte eignen sich die \textit{HOG-SVM}-Kombination und das quantisierte \textit{MobileNet V1 SSD} Netz für beide Hardwareplattformen nicht. Die \textit{MobileNet V1 SSD} Architektur zeigte die besten Ergebnisse hinsichtlich der Genauigkeit. Das \textit{MobileNet V2 SSDLite} Netz verarbeitet die Bilder der Benchmark am schnellsten. Je nach Anwendungsfall und Hardwareplattform werden die genannten KNNs angewendet.     \\

Der Personendetektion untergeordnet, wurde eine Gesichtsdetektion und -erkennung implementiert. Diese sucht bei einer Personendetektion im entsprechenden Interessensbereich nach vorhandenen Gesichtern. Wird ein Gesicht detektiert, erfolgt eine Merkmalsextraktion. Gesichter dienen hierbei als langfristig einzigartiges Unterscheidungsmerkmal von Personen. Nach der Merkmalsextraktion werden im System bekannte Personen wiedererkannt oder durch ein Registrierungsprozess in eine Datenbank eingepflegt.\\
\newpage
Der entwickelte Zustandsautomat steuert das Fahrzeug mithilfe der Ausgabe einer Sprachverarbeitung. Diese klassifiziert vom Benutzer eingesprochene, anwendungsorientierte Sprache. Die Klassifikation wird als Übergabewert und somit zur Eingabe des Zustandsautomats verwendet. Der EA ist durch seinen hierarchischen Aufbau in der Lage, Fahraufgaben in Form von Zuständen nacheinander aufzubauen. Weiterhin könnten in Zukunft vollständige Logistikanwendungen mithilfe dieses Systems realisiert werden. Eine Weiterentwicklung des Zustandsautomats ist möglich, indem ein übergeordnetes System implementiert wird. Dieses könnte Fahraufgaben gesondert klassifizieren. So kann zum Beispiel die Fahraufgabe \textit{Ziel} durch einen Anwendungsfall \textit{Paket abholen} oder auch \textit{Ware ausliefern} in Anspruch genommen werden.\\

In Kapitel \ref{ch: Einleitung} wurde als Beispiel die Freischaltung des ALFs für autorisierte Personen oder eine gesonderte Gefahreneinschätzung genannt. Letzteres könnte in Verbindung mit dem \textit{Robot Operating System} dazu führen, dass das Fahrzeug einen besonders großen Abstand zu Personen hält oder Bereiche mit hohem Personenaufkommen meidet. Als Hilfestellung und Grundlage für ein derartiges Projekt können die aus der Personenerkennung extrahierten Informationen dienen, wie zum Beispiel die Positionsschätzung aus Abschnitt \ref{sec: Bestimmung der Positionskoordinaten}. \\

Diesbezüglich ist ebenfalls die Entwicklung einer Personenverfolgung möglich. Hierbei könnten Laufwege von Personen sowohl analysiert als auch prädiziert werden. Entsprechende Informationen könnten wiederum auf die Routenplanung der Navigation einwirken, sodass beliebte Laufwege von dem ALF nicht oder selten befahren werden. \\

Bezüglich der festgestellten Werte der Bildverarbeitungsalgorithmen in Kapitel \ref{ch: Verifikation} sind weitere Optimierungen möglich. Der in dieser Masterarbeit erstellte Datensatz kann mit Bildern passend zum Einsatzort des ALFs erweitert werden. Eine Steigerung der Genauigkeit ist hierbei zu erwarten.\\

Weiterhin sollte insbesondere der Einsatz der Bildverarbeitungssysteme auf verschiedenen eingebetteten Systemen untersucht werden. Hierbei könnten unterschiedliche Hardwareplattformen in der selben Preiskategorie einer Benchmark unterzogen werden. Für ein derartiges Projekt würden zum Beispiel ein System des Typs \textit{BeagleBone} der Firma \textit{BeagleBoard.org} oder ein \textit{Jetson Nano} von \textit{NVIDIA} in Frage kommen. Das Ziel dieser Untersuchung sollte die Qualität der Objekterkennung hinsichtlich der Rechengeschwindigkeit steigern. 





