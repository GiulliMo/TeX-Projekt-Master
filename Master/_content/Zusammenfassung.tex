\chapter{Zusammenfassung und Ausblick}
\label{Fazit und Ausblick}
	
Im Rahmen dieser Masterarbeit wurde eine Personenerkennung für die Anwendung am autonomen Logistikfahrzeug ALF entwickelt und evaluiert. Die entwickelte Software ist darauf optimiert eine ausgewogene Balance zwischen einer hohen Genauigkeit bei kurzer Rechenzeit zu erreichen. Dies wird durch eine Unterteilung des Gesamtsystems in Teilsysteme auf verschiedenen Ebenen erreicht. Implementierte Softwarekomponenten eignen sich für den Einsatz auf eingebetteten Systemen und wurden bewusst danach ausgelegt. Eine dem Erkennungssystem übergeordnete Personendetektion erkennt Menschen im unmittelbaren Sichtkegel der verbauten \textit{Kinect}-Kameras. Die Bildverarbeitung nutzt ein künstliches neuronales Netzwerk zur Analyse von Bildern. Hierbei liegt das Augenmerk eher auf der Geschwindigkeit des Systems. Für die Auswahl der Netze wurden state-of-the-art Lösungen verglichen und mögliche Netztypen evaluiert.\\

%Das ausgewählte KNN zeigte in der in Kapitel \ref{ch: Verifikation} dargestellten Evaluation ein \textit{mAP}-Wert von... . Im Bezug auf die Geschwindigkeit der Verarbeitung eines Bildes innerhalb von circa ... ms zeigt dieses Netz im Vergleich die besten Vorraussetzungen.
Je nach Anwendungsfall können die Ergebnisse nach verschiedenen Gesichtspunkten evaluiert werden....\\

Der Personendetektion untergeordnet, wurde eine Gesichtsdetektion und -erkennung implementiert. Diese sucht bei einer Personendetektion im enstsprechenden Interessensbereich nach vorhandenen Gesichtern. Wird ein Gesicht detektiert erfolgt eine Merkmalsextraktion. Gesichter dienen hierbei als langfristig einzigartiges Unterscheidungsmerkmal von Personen. Nach der Merkmalsextraktion werden im System bekannte Personen wiedererkannt oder durch ein Registrierungsprozess in eine Datenbank eingepflegt.\\

Der entwickelte Zustandsautomat steuert das Fahrzeug ALF durch die Ausgabe einer Sprachverarbeitung. Diese klassifiziert vom Benutzer eingesprochene, anwendungsorientierte Sprache. Die Klassifikation wird als Übergabewert und somit zur Eingabe des Zustandsautomats verwendet. Der EA ist durch seinen hierarchischen Aufbau in der Lage Fahraufgaben in Form von Zuständen nacheinander aufzubauen. Weiterhin könnten in Zukunft vollständige Logistikanwendungen mithilfe dieses Systems realisiert werden. Eine Weiterentwicklung des Zustandsautomats ist möglich, indem zum Beispiel ein übergeordnetes System implementiert wird. Dieses könnte beispielsweise Fahraufgaben gesondert klassifizieren. So kann zum beispiel die Fahraufgabe \textit{Ziel} durch ein Anwendungsfall wie zum Beispiel \textit{Paket abholen} oder auch \textit{Paket ausliefern} in Anspruch genommen werden.\\


Zukünftige Entwicklungsarbeiten am ALF können sich an einem breiten Themenspektrum bedienen. In Kapitel \ref{ch: Einleitung} wurde als Beispiel die Freischaltung des ALFs für autorisierte Personen oder eine gesonderte Gefahreneinschätzung genannt. Letzteres könnte in Verbindung mit dem \textit{Robot Operating System} dazu führen, dass das Fahrzeug einen besonders großen Abstand zu Personen hält oder Bereiche mit hohem Personenaufkommen meidet. Als Hilfestellung und Grundlage für ein derartiges Projekt können die aus der Personenerkennung extrahierten Informationen dienen. \\

Bezüglich der festgestellten Werte der Bildverarbeitungsalgorithmen in Kapitel \ref{ch: Verifikation} sind weitere Optimierungen möglich. \textit{Tensorflow Lite} bietet beispielsweise die Möglichkeit Netze einer sogenannten Quantisierung zu unterziehen \cite{tflite}. Dies ermöglicht einen noch kleineren FPS-Wert bei geringer Einsparung der Genauigkeit. Weiterhin kann der in dieser Masterarbeit erstellte Datensatz mit Bildern passend zum Einsatzort des ALFs erweitert werden. Eine Steigerung der Genauigkeit ist hierbei zu erwarten.\\





