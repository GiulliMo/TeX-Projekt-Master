
\begin{figure}[H]
\begin{center}
	\begin{tikzpicture}[circuit ee IEC, circuit symbol lines/.style={draw,thick}]
	
	%Matrix
	\def\spaltenabstand{1.5cm}
	\def\zeilenabstand{0.5cm}
	\def\zeilenzahl{9}
	\def\spaltenzahl{7}
	\foreach \i in {1,...,\zeilenzahl}
	\foreach \j in {1,...,\spaltenzahl}
	{
		\coordinate (S-\i-\j) at ({(\j-1)*\spaltenabstand},{-(\i-1)*\zeilenabstand});
		%\node at (S-\i-\j){+}; % Orientierungshilfe
	}
	
	%Schaltung
	\draw (S-2-1) to  [resistor={info=$R_1$}](S-4-1); 
	\draw (S-6-1) to  [inductor={info=$L_1$}](S-8-1); 
	%\draw (S-4-2)[fill=white] circle [radius=15pt];
	\node[draw,circle,minimum size=1cm,inner sep=0pt] at (S-5-2){$L CR$};
	%Leiterbahnen
	
	\draw 
	(S-2-1)--(S-1-1)--(S-1-2)--(S-4-2)
	(S-6-1)--(S-4-1)
	(S-8-1)--(S-9-1)--(S-9-2)--(S-6-2);
	
	%Schaltungsaufbau, oben
	%\draw (S-1-3) to  [inductor={info=$L_1$}](S-1-4.east); 
	%\draw (S-3-3) to  [capacitor={info=$C_1$}](S-3-4);
	
	%Schaltungsaufbau, unten
	%\draw[set inductor graphic=var inductor IEC graphic]
%	(S-5-3) to [inductor={info=$L_2$}](S-5-4);
%	\draw (S-7-3) to  [capacitor={info=$C_2$}](S-7-4); 
%	\draw (S-6-4) to  [capacitor={info=$C_3$}](S-6-5);
	
	
	Knotenpunkte, Klemmen (extra zeichnen, so keine Überschneidungen)
	\foreach \p in {S-1-1,S-9-1}
	\draw[fill=white] (\p) circle [radius=2pt];
	
	%\foreach \p in {S-4-2,S-2-3,S-2-4,S-4-5,S-6-3,S-6-4}
	%\draw[fill] (\p) circle [radius=2pt];
	
	\end{tikzpicture}
\end{center}
\centering
\caption{Versuchsaufbau Messung der Indktivität}
\label{fig:Versuchsaufbau Messung der Indktivität}
\end{figure}