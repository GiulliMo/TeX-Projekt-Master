%	Autor:		Sean Dalton	
%	Datum:		tt.mm.jjjj
%	Short:		Bachelorarbeit
% DOCUMENTCLASS -----------------------------------------------------------------------------------------
\documentclass[12pt,a4paper,oneside,numbers=noenddot,headsepline,captions=tableheading,toc=bibliography,openany]{scrbook}

\usepackage{geometry}
\geometry{a4paper,left=40mm,right=25mm, top=25mm, bottom=25mm, includeheadfoot}


% SPRACHE------------------------------------------------------------------------------------------------
	\usepackage[utf8]{inputenc}
	\usepackage[autostyle,babel,german=guillemets,style=german]{csquotes}
	\usepackage[USenglish,ngerman]{babel} 
	\usepackage{lmodern} %Hochaufloesende Schrift
	\selectlanguage{ngerman}
	

% TABELLEN-----------------------------------------------------------------------------------------------
	\usepackage{multirow} % Tabellen-Zellen �ber mehrere Zeilen1
	\usepackage{multicol} % mehre Spalten auf eine Seite
	\usepackage{tabularx} % F�r Tabellen mit vorgegeben Gr��en
	\usepackage{array}
	\usepackage{float}
	\usepackage{booktabs}
	\usepackage{longtable} % Tabellen �ber mehrere Seiten
	\usepackage{rotating}
	\usepackage[table,xcdraw]{xcolor}
	

	\usepackage{pdflscape}
	
% BILDER-------------------------------------------------------------------------------------------------
	\usepackage{graphicx}
	\usepackage{color}


% ALLGEMEINES--------------------------------------------------------------------------------------------
	\usepackage{amsmath,amssymb} % Mathesachen
	\usepackage[T1]{fontenc} % Ligaturen, richtige Umlaute im PDF
	\usepackage[hyphens,obeyspaces,spaces]{url}		% bricht lange URLs ?sch�n? um
	\usepackage[decimalsymbol=comma]{siunitx}
	\usepackage[onehalfspacing]{setspace}
	\setlength{\parindent}{0cm}	% Disable automatic parindent
	\usepackage{placeins}
	\sloppy
	
	\usepackage{pdfpages}
	\usepackage{enumitem}
	\graphicspath{{_img/}}
% PDF-----------------------------------------------------------------------------------------------------
	\usepackage{pdfpages}
	\usepackage[ngerman,%
	pdfauthor={Sean Dalton},%
	pdftitle={Entwicklung der eingebetteten Hardware einer modularen Antriebsplattform},%
	colorlinks=true,linkcolor=black,citecolor=black,filecolor=black,urlcolor=black%
	]{hyperref}

% ABKÜRZUNGEN --------------------------------------------------------------------------------------------
	\usepackage[nohyperlinks]{acronym}


% BIBLATEX -----------------------------------------------------------------------------------------------
	\usepackage[backend=bibtex,%
	style=IEEEtran,%
	bibstyle=numeric,%
	citestyle=numeric,%
	sorting=none,%
	]{biblatex}
	\bibliography{_bib/biblio}
	
% QUELLCODE -------------------------------------------------------------------
\usepackage{listings}
\definecolor{mygreen}{RGB}{28,172,0} 		% color values Red, Green, Blue
\definecolor{mylilas}{RGB}{170,55,241}
\lstset{language=Matlab,%
	basicstyle=\footnotesize\ttfamily,
	breaklines=true,%
	morekeywords={matlab2tikz},
	keywordstyle=\color{blue},%
	morekeywords=[2]{1}, keywordstyle=[2]{\color{black}},
	identifierstyle=\color{black},%
	stringstyle=\color{mylilas},
	commentstyle=\color{mygreen},%
	showstringspaces=false,%without this there will be a symbol in the places where there is a space
	numbers=left,%
	numberstyle={\tiny \color{black}},% size of the numbers
	numbersep=9pt, % this defines how far the numbers are from the text
	emph=[1]{for,end,break},emphstyle=[1]\color{red}, %some words to emphasise
	%emph=[2]{word1,word2}, emphstyle=[2]{style},    
}	
	
% DOCUMENT ------------------------------------------------------------------------------------------------
\begin{document}

%\pagestyle{empty}


\cleardoublepage

% ----------------------------------------------------------------------------------------------------------
% TITELSEITE -----------------------------------------------------------------------------------------------
% ----------------------------------------------------------------------------------------------------------
\begin{titlepage}
	\singlespacing
	
%----------------Einzelnes Logo	
	\begin{center}
		\includegraphics[width=0.5\textwidth]{logo}
	\end{center}

%----------------Zwei Logo		
%	\begin{figure}
%		\centering
%		\begin{minipage}[b]{0.5\textwidth}
%			\includegraphics[width=\textwidth]{_img/Logo_Smart_RGB}
%		\end{minipage}
%		\hfill
%		\begin{minipage}[b]{0.4\textwidth}
%			\includegraphics[width=\textwidth]{_img/logo}
%		\end{minipage}
%	\end{figure}
	
	\begin{center}
		\vspace*{3cm}
		\Huge
		\textbf{Entwicklungsprojekt}\\
		\vspace*{0.5cm}
		\large
		zur Erlangung des akademischen Grades\\
		Bachelor of Engineering (B.Eng.)\\
		\vspace*{1cm}
		\huge
		\textbf{Entwicklung der eingebetteten Hardware einer modularen Antriebsplattform }\\
		\vspace*{2cm}
		\vfill
		\normalsize
		\newcolumntype{x}[1]{>{\raggedleft\arraybackslash\hspace{0pt}}p{#1}}
		
		\begin{tabular}{x{6cm}p{7.5cm}}
			\rule{0mm}{5ex}\textbf{Autor:} & Sean William Dalton\newline sean.dalton@hs-bochum.de\newline Matrikelnummer: 013208208\\
			\rule{0mm}{5ex}\textbf{Erstgutachter:} & Prof.\ Dr.-Ing.\ Arno Bergmann \\
			\rule{0mm}{2ex}\textbf{Zweitgutachter:} &  \\ 
			\rule{0mm}{5ex}\textbf{Abgabedatum:} & tt.mm.jjjj\\ 
		\end{tabular} 
	\end{center}
\end{titlepage}
\onehalfspacing
\cleardoublepage
%\input{_vermerke/erklaerung}
\cleardoublepage
% ----------------------------------------------------------------------------------------------------------
% VERZEICHNISSE --------------------------------------------------------------------------------------------
% ----------------------------------------------------------------------------------------------------------
 \pagenumbering{roman}
 
 \tableofcontents
 \addcontentsline{toc}{chapter}{Inhaltsverzeichnis}
 \cleardoublepage

\chapter*{Abkürzungsverzeichnis}
\addcontentsline{toc}{chapter}{Abkürzungsverzeichnis}
\begin{acronym}[PMSM]	% längste Abkürzung
	\acro{ASM}{Asynchronmaschine}
	\acro{GSM}{Gleichstrommaschine}
	\acro{PMSM}{Permanentmagneterregte Synchronmaschine}	
\end{acronym}
\cleardoublepage

\pagestyle{plain}

\chapter*{Symbolverzeichnis}
\addcontentsline{toc}{chapter}{Symbolverzeichnis}

\begin{longtable}{p{3cm}p{8cm}p{2cm}}
	\setlength\tabcolsep{9pt}
Symbol & Bedeutung                & Einheit               \\ \hline
	\endfirsthead
Symbol             & Bedeutung                & Einheit               \\ \hline
	\endhead
	\toprule
	$B$			& magnetische Flussdichte				& \si{\tesla} \\
	$D$			& Elektrische Flussdichte				& \si{\ampere\second\per\square\meter} \\
$E$			& Elektrische Feldstärke				& \si{\volt\per\meter} \\
$H$			& magnetische Feldstärke				& \si{\ampere\per\meter} \\

\bottomrule
\end{longtable}


\cleardoublepage
\pagestyle{headings}
\pagenumbering{arabic}
\setlength{\parskip}{0.1em}
% ----------------------------------------------------------------------------------------------------------
% INHALT ---------------------------------------------------------------------------------------------------
% ----------------------------------------------------------------------------------------------------------

% -*- coding: utf-8 -*-
% !TEX encoding = UTF-8 Unicode
% !TEX root =  main.tex

\chapter{Beispiel Kapitel}
\label{chap:beispiel-kapitel}

In diesem Kapitel wird eine sehr kurze Einleitung in die Verwendung von \verb|LaTeX| beschrieben.\\
Für die Erstellung der Arbeit kann über \url{http://www.texstudio.org/} mit installierter TeX-Distribution \url{https://miktex.org/} das TeXstudio heruntergeladen werden. 

\section{Abbildungen}
Eine Abbildung lässt sich einfach über einfügen: \par\medskip

\lstset{language=TeX}
\begin{lstlisting}
\begin{figure}[h]
\centering
\includegraphics[width=\textwidth]{foc-ac-dc.pdf}
\caption{Beschriftung der Abbildung}
\label{fig:foc-ac-dc}
\end{figure}
\FloatBarrier
\end{lstlisting}

Die breite der Abbildung kann einerseits skaliert oder direkt im Maßstab von \SI{14.5}{\centi\meter} erstellt werden.
Wenn die Abbildung maßstabsgetreu erstellt wird, muss \verb|\centering| und der optionale Befehl \verb|[width=\textwidth]| nicht zwingend übernommen werden.\par\medskip

Es werden beim Auftreten des Befehls \verb|\FloatBarrier| alle bis dahin eingefügten Float-Umgebungen gesetzt Das kann z.B. dazu verwendet werden, dass Floats, wie figure oder table, nicht unterhalb einer neuen \verb|section| oder \verb|chapter| ausgegeben werden.\par\medskip

Abbildung \ref{fig:foc-dc-ac} zeigt \ldots \par\medskip

\begin{figure}[h]
	\centering
	\includegraphics[width=\textwidth]{foc-dc-ac.pdf}
	\caption{Beschriftung der Abbildung}
	\label{fig:foc-dc-ac}
\end{figure}
\FloatBarrier

Durch setzten des \verb|[h]| hinter der \verb|figure| Umgebung, kann die Positionierung der Abbildung festgelegt werden.\par\medskip

Dabei sind folgende Werte ebenfalls möglich:

\begin{enumerate}
	\item h (here) - Gleicher Ort
	\item t (top) - Oben auf der Seite
	\item b (bottom) - Unten auf der Seite
	\item p (page) - Auf einer eigenen Seite
	\item ! (override) - Erzwingt die angegebene Position
\end{enumerate}

\section{Tabellen}\label{sec:tab}

Zur Erstellung einfacher Tabellen, bietet die Website \url{http://www.tablesgenerator.com/} eine einfach zu bedienende Oberfläche. \par\medskip %erzeugt einen mittleren Abstand

Für komplexere Tabellen können die \verb|multicol| und \verb|multirow| Pakete verwendet werden, wie in Tabelle \ref{tab:vgl_mosfet} dargestellt \cite{dalton}.
\begin{table}[h]
		\centering
	\renewcommand{\arraystretch}{1.6}
\begin{tabular}{lc|c|c|c|l}
	\cline{3-5}
	& \multicolumn{1}{l|}{} & \multicolumn{3}{c|}{MOSFET}             &  \\ \cline{3-5}
	& \multicolumn{1}{l|}{} & IRFS7530       & IPB019N08N3 & CSD19536KCS &  \\ \cline{3-5}
	& \multicolumn{1}{l|}{} & International Rectifier       & Infineon & Texas Instruments &  \\ \cline{1-5}
	
	\multicolumn{1}{|l|}{\multirow{4}{*}{\begin{turn}{90}Parameter\end{turn}}} & $Q_{\mathsf{G}}$                    &   354 \nano\coulomb             &  206 \nano\coulomb           &   153 \nano\coulomb      &  \\ \cline{2-5}
	\multicolumn{1}{|l|}{}                                                           & $Q_{\mathsf{GS}}$                   &  62 \nano\coulomb             &  50 \nano\coulomb          &  37 \nano\coulomb       &  \\ \cline{2-5}
	\multicolumn{1}{|l|}{}                                                           & $Q_{\mathsf{GD}}$                   & 73 \nano\coulomb              &  30 \nano\coulomb           &  17 \nano\coulomb       &  \\ \cline{2-5}
	\multicolumn{1}{|l|}{}                                                           & $R_{\mathsf{DSon}}$                 & 1,4 \milli\ohm & 1,9 \milli\ohm            &  3,5 \milli\ohm        &  \\ \cline{1-5}
\end{tabular}
	
	\caption{Vergleich verschiedener MOSFET \cite{dalton}}
	\label{tab:vgl_mosfet}
\end{table}
\FloatBarrier

\section{Zitate}\label{sec:cite}
Für Zitationen wird \verb|BibLaTeX| verwendet. Als Backend wird \verb|bibtex| vom Compiler verlangt.


\begin{quote}
\enquote{Bei jeder permanentmagneterregten Synchronmaschine ändern sich die Induktivitäten in Abhängigkeit von der Last. In erster Linie sind dafür die Sättigungseffekte, aber auch die Kreuzkopplung verantwortlich.} 
\end{quote}

Im Text zitierte Werke werden über die Syntax \verb|\textcite[S.~2]{ternes2015}| korrekt zitiert. Beispielsweise: Wie in \cite{ternes2015} erläutert, sind die Induktivitäten abhängig von der Last \ldots   \par\medskip

Der aktuelle Stil des Literaturverzeichnisses und der Zitationen ist \verb|IEEEtran|, kann aber auch in Absprache geändert werden, dazu empfiehlt es sich, die \verb|BibLaTeX|-Dokumentation zu konsultieren.\\
\clearpage
\section{Anhänge}\label{sec:Anhang}
Um Anhänge zu referenzieren, können diese mit Hilfe des  erstellten Anhangs (vgl. \ref{anhang_LastenheftEpOS}) referenziert werden.

\section{Formeln}\label{sec:Formeln}

Bei Implementierung von Formeln mit eingesetzten Werten, erweist sich eine Kombination von Tabelle und Formel als Sinnvoll. Tabelle \ref{tab:param_voltageDrop} zeigt die in Formel \ref{eq:voltagedrop} eingesetzten Parameter, mit Beschreibung sowie dem zugehörigen Wert \cite{drv8303}. \par\medskip

\begin{table}[h]
	\centering
	\begin{tabular}{ccc}
		\hline
		Parameter        &       Beschreibung        &           Wert           \\ \hline
		$V_{\mathsf{0}}$    & Maximale Ausgangsspannung &        3,3\ \volt        \\
		$V_{\mathsf{REF}}$   &     Referenz-Spannung     &        3,3\ \volt        \\
		$G$           &    Verstärkungsfaktor     & 40 $\frac{\volt}{\volt}$ \\
		${R}_{\mathsf{SHUNT}}$ &     Shunt-Widerstand      &    $500\ \micro\ohm$     \\ \hline
	\end{tabular}
	\caption{Parameter der Operationsverstärker-Einstellungen des DRV8303}
	\label{tab:param_voltageDrop}
\end{table}


\begin{equation}
\centering
V_{\mathsf{Smax}} = {\lvert ({SN}_{\mathsf{x}} - {SP}_{\mathsf{x}}) \rvert}_{\mathsf{max}} = \left( \frac{ \left({V}_{0} - \frac{ {V}_{\mathsf{REF}}}{2} \right)}{G}\right)   = 41,25\ \milli \volt
\label{eq:voltagedrop}
\end{equation}
\FloatBarrier

%%% Local Variables: 
%%% mode: latex
%%% TeX-master: "main"
%%% TeX-open-quote: "\\enquote{"
%%% TeX-close-quote: "}"
%%% LaTeX-csquotes-open-quote: "\\enquote{"
%%% LaTeX-csquotes-close-quote: "}"
%%% End: 

%\input{_content/einleitung}
%\input{_content/grundlagen}
%\input{_content/systemkonzipierung}
%\input{_content/implementierung}
%\input{_content/verifikation}
%\input{_content/fazit}
\clearpage

   
%Almost done
\pagenumbering{Roman}
\clearpage

% ----------------------------------------------------------------------------------------------------------
% REFERENZEN -----------------------------------------------------------------------------------------------
% ----------------------------------------------------------------------------------------------------------
\listoffigures
\addcontentsline{toc}{chapter}{Abbildungsverzeichnis}
\clearpage

\listoftables
\addcontentsline{toc}{chapter}{Tabellenverzeichnis}
\clearpage

%\printbibliography
\clearpage


%Ende der Arbeit -----------------------------------------------------------------------
\end{document}